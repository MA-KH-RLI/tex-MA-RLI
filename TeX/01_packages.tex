%%%%%%%%% Dokumentenklasse
\documentclass[
	%fontsize=12pt,% Times New Roman
	fontsize=11pt,% Andere
	paper=a4,
	]{scrartcl}

% \flushbottom % extends white space between paragraphs to fill pages

%%%%%%%%% HTW Anforderungen
\usepackage[
	margin=2.5cm,% Seitenränder auf 2,5cm setzen
%1	bottom=3.1cm,
%	showframe,% <- only to show the page layout
]{geometry}
%\usepackage{mathptmx}% Times New Roman
%\usepackage[scaled=0.92]{helvet}% quasi Arial
%\renewcommand{\familydefault}{\sfdefault}
%\usepackage{roboto}% Roboto Font


%%%%%%%%% Literatur
\usepackage[
	backend = biber,
	style=alphabetic,% Alphatischer Style [abc12]
	maxalphanames=3,% Kurzform besteht aus maximal den ersten Buchstaben der Nachnamen der ersten drei Autoren (wenn min. 3 gegeben)
	minalphanames=3,% Kurzform besteht aus minimal drei Buchstaben der Nachnamen der Autoren
	]{biblatex}% Nützliche Anleitung: http://www.nagel-net.de/Latex/DOKU/DTK-2_2008-biblatex-Teil1.pdf
\usepackage{notoccite}% don't show cites in toc, list of figures, etc.


%%%%%%%%% Code
\usepackage[
	newfloat,
%	outputdir=out,
%	cache=false,
	]{minted}% Code highlighting
\usepackage{caption}% Neue Environment um Code-Blöcken eine Caption zu geben
\newenvironment{code}{\captionsetup{type=listing}}{}
\SetupFloatingEnvironment{listing}{name=Quellcode}


%%%%%%%%% Sprachpackages
\usepackage[
	T1
	]{fontenc}% Textzeichen für westeuropäische Sprachen
\usepackage[
	utf8
	]{inputenc}% Umlaute
\usepackage[
	ngerman,% Deutsch
	english,% Englisch
	]{babel}% Automatisch erzeugte Texte werden auf Deutsch ausgegeben
\usepackage[
	ngerman
	]{datetime2}% Datumsformate
\usepackage{csquotes}% This package provides advanced facilities for inline and display quotations; should be loaded after minted
\usepackage{textcomp}% Einige Symbole
\usepackage[
	official
	]{eurosym}% €-Symbol


%%%%%%%%% Blind-Text
\usepackage{lipsum}
\usepackage{draftwatermark}

\SetWatermarkText{Draft: \today}
\SetWatermarkColor[gray]{0.5}
\SetWatermarkFontSize{1cm}
\SetWatermarkAngle{90}
\SetWatermarkHorCenter{20cm}


%%%%%%%%% Mathe & Naturwissenschaften
\usepackage[
%	fleqn
	]{amsmath}% Mathematische Formartierung
\usepackage{
	amssymb,% Mathematische Symbole
	amsfonts,% Mathematische Schriftarten
	}
\usepackage{siunitx}% Saubere Darestellung von SI Einheiten
\sisetup{
	locale = DE,% Deutsche Norm, Kommas werden z.B. erkannt
	per-mode=symbol,% Output a/b as \frac{a}{b} - in der Einheit
	quotient-mode=symbol,% Output a/b as \frac{a}{b} - im Quotienten
	fraction-function=\tfrac,
	range-phrase = {\text{~bis~}},
	sticky-per = true,% \per bleibt bestehen für mehr als eine Einheit
	separate-uncertainty,% Standardabweichung
}
\usepackage{mathtools}
\usepackage[
	version=4,
	]{mhchem}% chemische Gleichungen/Summenformeln


%%%%%%%%% Grafiken und Farben
\usepackage{graphicx}% viele grafische Befehle, z.B. \scalebox
\usepackage{
	xcolor,
	colortbl,
	}% Farben und Farbpaletten
\usepackage[
	export
	]{adjustbox}% Positionieren von Grafiken, left, right, center
\usepackage{float}% Floating Bilder, H-Befehl
\usepackage{mwe}% Für Abbildungsverzeichnis
\usepackage{graphics}% accommodate all needs for inclusion of graphics
\usepackage{subfigure}% support for the manipulation and reference of small or ‘sub’ figures and tables within a single figure or table environment
\usepackage{svg}% einbinden von svg Grafiken


%%%%%%%%% Diagramme
\usepackage{
	tikz,% Grafik-Paket
	stackengine,% versatile way to stack objects vertically in a variety of customizable ways
	}
\usepackage{pgfplots}% Plots
\pgfplotsset{compat=1.14}% Soll man wohl machen
\usetikzlibrary{patterns}% Patterns anstatt Farben


%%%%%%%%% Seitenformatierung
\makeatletter
\usepackage{microtype}% Verbesserte Formatierung; Sollte immer geladen werden
\g@addto@macro\@verbatim{\microtypesetup{activate=false}}
\makeatother
\usepackage[
	headsepline,% Vertikale Linie unterm Header
	plainheadsepline,
	automark,% Section im Header
	singlespacing=true,
	]{scrlayer-scrpage}% Paket zur Manipulation der Kopf- und Fußzeilen
\usepackage{adjustbox}% Elemente skalieren \scalebox
\usepackage{pdflscape}% ermöglicht einzelne Seiten im landscape mode darzustellen
\usepackage{abstract}% Abstract-Umgebung
\usepackage{setspace}% Spacing-Umgebung
\usepackage{verbatimbox}
\usepackage[htt]{hyphenat}% better linebreaks in long \texttt{•} etc.


%%%%%%%%% Tabellenumgebung
\usepackage{booktabs}% enhances the quality of tables
\usepackage{array}% extends the options for column formats
\usepackage{
	multirow,
	makecell
	}% mehrzeilige Tabellenzellen
\usepackage{tabu}% Moderneres tabularx


%%%%%%%%% Custom
\newenvironment{conditions}% New environment - Für saubere Darstellung gegebener Variablen
	{\par\vspace{\abovedisplayskip}\noindent\begin{tabular}{>{$}l<{$} @{${}={}$} l}}
	{\end{tabular}\par\vspace{\belowdisplayskip}}
\makeatletter% Roman Numbers in Text
\newcommand*{\rom}[1]{\expandafter\@slowromancap\romannumeral #1@}
\makeatother
\renewcommand*{\labelalphaothers}{\textsuperscript{+}}% superscript + instead of normal + in literature


%%%%%%%%% Querverweise
\usepackage[
	breaklinks,% URLs/DOIs vernünftig brechen
	]{hyperref}% Links im pdf % Muss das letzte Paket sein was lädt, außer glossaries
\hypersetup{
    colorlinks,
    linkcolor={red!40!black},
    citecolor={blue!50!black},
    urlcolor={blue!80!black}
}


% Kapitel, statt Abschnitt, Unterabschnitt, etc.
\addto\extrasngerman{
	\def
	\sectionautorefname{Kapitel}
}
\addto\extrasngerman{
	\def
	\subsectionautorefname{Kapitel}
}
\addto\extrasngerman{
	\def
	\subsubsectionautorefname{Kapitel}
}


%%%%%%%%% Abkürzungsverzeichnis:
\usepackage[% Paket für Glossaries und Acronym-Glossaries % Muss das letzte Paket sein was lädt
	acronym,
	automake,
	nopostdot,
	toc,
	nomain,
	shortcuts,
	nogroupskip,
	]{glossaries}


%%%%%%%%% Date Range

\renewcommand{\DTMdisplaydate}[6]{\DTMtwodigits{#1}.\DTMtwodigits{#2}.~{--}~\DTMtwodigits{#4}.\DTMtwodigits{#5}.}


%%%%%%%%% Sonstige Custom Befehle:
\def\Arbeit{\glqq Arbeit\grqq{}~}
\def\Arbeitdot{\glqq Arbeit\grqq}
\def\Ausbildung{\glqq Ausbildung\grqq{}~}
\def\dienst{\glqq dienstlich\grqq{}~}
\def\dienstdot{\glqq dienstlich\grqq}
\def\ego{\textit{eGo~100}~}
\def\egodot{\textit{eGo~100}}
\def\Eigenheim{\glqq Eigenheim\grqq{}~}
\def\Eigenheimdot{\glqq Eigenheim\grqq}
\def\Einkauf{\glqq Einkauf\grqq{}~}
\def\Einkaufdot{\glqq Einkauf\grqq}
\def\Erledigung{\glqq Erledigung\grqq{}~}
\def\Erledigungdot{\glqq Erledigung\grqq}
\def\Firmeparkplatz{\glqq Firmenparkplatz\grqq{}~}
\def\Firmeparkplatzdot{\glqq Firmenparkplatz\grqq}
\def\Firmeparkplaetzen{\glqq Firmenparkplätzen\grqq{}~}
\def\Firmeparkplaetzendot{\glqq Firmenparkplätzen\grqq{}}
\def\Freizeit{\glqq Freizeit\grqq{}~}
\def\Freizeitdot{\glqq Freizeit\grqq}
\def\Gewerbeparkplatz{\glqq Gewerbeparkplatz\grqq{}~}
\def\Gewerbeparkplatzdot{\glqq Gewerbeparkplatz\grqq}
\def\Kleinwagen{Szenarette \glqq Kleinwagen\grqq{}~}
\def\Kleinwagendot{Szenarette \glqq Kleinwagen\grqq}
\def\kmean{\textit{k-means-Clustering}~}
\def\kmeans{\textit{k-means-Clusterings}~}
\def\kmeansdot{\textit{k-means-Clusterings}}
\def\Lastgebiet{\glqq Lastgebiet\grqq{}~}
\def\Lastgebiete{\glqq Lastgebiete\grqq{}~}
\def\Lastgebietedot{\glqq Lastgebiete\grqq}
\def\Lastgebietes{\glqq Lastgebietes\grqq{}~}
\def\Lastgebieten{\glqq Lastgebieten\grqq{}~}
\def\nH{\glqq nach Hause\grqq{}~}
\def\nHdot{\glqq nach Hause\grqq}
\def\oeffen{\glqq Öffentlich\grqq{}~}
\def\oeffendot{\glqq Öffentlich\grqq}
\def\Plus{\texttt{+}}
\def\Straszenrand{\glqq Straßenrand\grqq{}~}
\def\Straszenranddot{\glqq Straßenrand\grqq}
\def\SzeFirmenparkplatz{Sensitivität \glqq Firmenparkplatz\grqq{}~}
\def\SzeFirmenparkplatzdot{Sensitivität \glqq Firmenparkplatz\grqq}
\def\UC{Lade~\textit{Use}~\textit{case}~}
\def\UCs{Lade~\textit{Use}~\textit{cases}~}
\def\Wohnanlage{\glqq Wohnanlage\grqq{}~}
\def\Wohnanlagedot{\glqq Wohnanlage\grqq}
\def\zH{\glqq zu Hause\grqq{}~}
\def\zHdot{\glqq zu Hause\grqq}