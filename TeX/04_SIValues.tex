% Shortcuts for SI Values
% Grundlagen: https://www.namsu.de/Extra/pakete/Siunitx.html
% Gute Erklärung der Shortcuts: https://texwelt.de/fragen/2588/wie-schreibe-ich-zahlen-mit-einheiten-richtig

%\DeclareSIUnit{\BeladungsDichte}{\kilo\gram_{\textup{H\textup{2}}\per\kilo\gram_{\textup{FeTi}}}}		% Beladungsdichte
%\DeclareDocumentCommand\BeladungsDichte{O{}m}{\SI[#1]{#2}{\BeladungsDichte}}

%\DeclareSIUnit[]\NormVolumen
%{\text{\ensuremath{\cubic\meter_{\textup{i.N.}}}}}

%%%%%%% New SIValues

\DeclareSIUnit\sieuro{\mbox{\euro{}}}
\DeclareSIUnit\kw{\kilo\watt}
\DeclareSIUnit\mw{\mega\watt}
\DeclareSIUnit\gw{\giga\watt}
\DeclareSIUnit\gwh{\giga\watt\hour}
\DeclareSIUnit\kv{kV}
\DeclareSIUnit\kva{kVA}
\DeclareSIUnit\mva{MVA}
\DeclareSIUnit\kwh{kWh}
\DeclareSIUnit\gwh{GWh}
\DeclareSIUnit\twh{TWh}
\DeclareSIUnit\kwhkm{\kwh\per100~\km}
\DeclareSIUnit\MioMen{\text{Millionen~Menschen}}
\DeclareSIUnit\MioStk{\text{Millionen~Stück}}
\DeclareSIUnit\MioStkSC{\text{Mio.~Stk.}}
\DeclareSIUnit\Minuten{\text{Minuten}}

%%%%%%% New complete Commands

\NewDocumentCommand\DeclareNewQuantity{mmm}{%
	\DeclareSIUnit{#2}{#3}%
	\DeclareDocumentCommand{#1}{O{}m}{\SI[##1]{##2}{#2}}%
}

\DeclareNewQuantity
	\Dichte
	\dichte
	{\kg\per\cubic\meter}