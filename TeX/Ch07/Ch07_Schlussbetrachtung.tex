\section{Schlussbetrachtung und Ausblick}\label{chap:schlussbetrachtung}

Das Ziel dieser Arbeit war es, die Auswirkungen verschiedener präventiver und aktiver Ladestrategien bei unterschiedlichen Durchdringungstiefen von \glspl{EPKW} auf die drei Netzgebietsklassen \gls{PV}-, Wind- und Last-dominiert aufzuzeigen.
Dafür wurde der jeweils nötige last- und erzeugerseitige Abregelungsbedarf innerhalb der untersuchten \gls{MS}-Netze, inklusive daruterliegender \gls{NS}-Netze, quantifiziert.\medskip

Es kann gezeigt werden, dass mit \gls{SIMBEV} ein Software Tool mitentwicklet wurde, mit dessen Hilfe plausible Fahrtprofile von \gls{EPKW} erzeugt werden können.
Die Fahrtprofile zeigen deutlich, dass ein Großteil der Ladevorgänge im privaten Bereich stattfindet.
Bei privaten Ladevorgängen wird weiterhin deutlich, dass die Ladung der \glspl{EPKW} nur einen geringen Anteil der Standzeit ausmacht.
In den untersuchten Szenarien entstehen die stärksten Leistungsspitezen am morgen durch das Lade am Arbeitsplatz, während das Laden zu Hause den größten Energiebedarf aufweist.

Die anschließende Verortung des Ladebedarfs führt teilweise zu unplausiblen Belastungen einzelner \gls{NS}-Netze innerhalb der Referenznetzgebiete.
In der Realität würde es hierbei vermutlich zu einem Netzneubau kommen, welcher innerhalb dieser Arbeit nicht abgebildet werden kann.\medskip

Bei den Ladestrategien wird nur mit Hilfe des Residuallast-Ladens eine wesentliche Glättung der Residuallast erreicht, während die präventiven Ladestrategien zu einer Erhöhung der Spreizung zwischen der maximalen und minimalen Residuallast führen.
Der negative Effekt der präventiven Ladestrategien lässt sich primär in Netzen mit einem hohen Anteil an \glspl{PVA} beobachten.
Am frühen Nachmittag kommt es in diesen Netzen aufgrund der präventiven Ladestrategien zu einem geringeren Ladebedarf.
Da in diesem Zeitfenster die Einspeisung der \glspl{PVA} in der Regel hoch ausfällt, wird hierdurch die Residuallast in diesen Zeitschritten gesenkt.\medskip

In den \gls{PV}-dominierten Netzen kommt es ausschließlich zu einer Abregelung von \gls{FEE} Anlagen in der Hochzeit der \gls{PV}-Einspeisung um die Mittagszeit herum.
Bei den präventiven Ladestrategien fällt in diesem Zeitfenster wie zuvor beschrieben der Ladebedarf geringer aus als beim Referenz-Laden, wovon vor allem der \UC \zH betroffen ist.
Da in \gls{NS}-Netzen mit einem hohen Anteil an Ladevorgängen des \UC \zH in der Regel auch viele \glspl{PVA} verortet sind, erhöht sich in diesen Fällen aufgrund der präventiven Ladestrategien der Abregelungsbedarf von \gls{FEE} Anlagen.
In den Wind-dominierten Netzen ist dieser Effekt aufgrund des niedrigen Anteils von \glspl{PVA} an den Erzeugerkapazitäten geringer ausgeprägt.\medskip

Durch das Residuallast-Laden kann der Abregelungsbedarf von \gls{FEE} Anlagen in \gls{PV}-dominierten Netzen gesenkt werden, da viele Ladevorgänge in die Mittagszeit verschoben werden und es zu einem besseren Ausgleich zwischen Last und Erzeugung kommt.
In den Wind-dominierten Netzen kommt es hingegen aufgrund des Residuallast-Ladens zu einem erhöhten Abregelungsbedarf an \gls{FEE} Anlagen.
In diesen Fällen lassen sich zwei Effekte beobachten.
Zum einen befindet sich in den Wind-dominierten Netzen zusätzlich ein gewisser Anteil an \glspl{PVA}, welcher wie zuvor beschrieben vermehrt in \gls{NS}-Netzen mit einem hohen Anteil an Ladevorgängen des \UC \zH verortet ist.
Aufgrund einer niedrigen Residuallast in der Nacht, werden jedoch viele Ladevorgänge des \UC \zH in die Nacht verschoben.
Zum anderen befinden sich Wind-Kapazitäten nicht in örtlicher Nähe der privaten Ladeinfrastruktur und werden auf der \gls{MS}-Ebene angeschlossen.
Hierdurch wird der Abregelungsbedarf der Wind-Kapazitäten durch die Ladestrategien kaum beeinflusst, während sich der Abregelungsbedarf der \glspl{PVA}, und somit in Summe der Abregelungsbedarf von \gls{FEE} Anlagen, erhöht.\medskip

Der lastseitige Abregelungsbedarf kann sowohl durch das reduzierte Laden als auch das Residuallast-Laden in den meisten Fällen gesenkt werden.
Dabei ist insbesondere der Erfolg des Residuallast-Ladens davon abhängig, ob sich die Erzeugerkapazitäten in den gleichen \gls{NS}-Netzen befinden wie die Ladestationen, ob die Ladung der \glspl{EPKW} und die Einspeisung innerhalb eines \gls{NS}-Netzes gleichzeitig erfolgen und ob die \gls{NS}-Netze nur in einer begrenzten Anzahl von Zeitschritten überlastet sind.
Sind diese Bedingungen gegeben, dann kann durch das Residuallast-Laden ein Großteil des lastseitigen Abregelungsbedarfs vermieden werden.
Bei den Wind-dominierten Netzen ist dies nicht der Fall, da sich das Residuallast-Laden an einer globalen Residuallast im gesamten Netzgebiet orientiert, welche nur bedingt die lokale Situation widerspiegelt.
In diesen Fällen kann durch das reduzierte Laden der lastseitige Abregelungsbedarf stärker gesenkt werden.\medskip

Mit einem steigenden Ladebedarf aufgrund eines erhöhten Hochlaufs der \glspl{EPKW} mit den Szenarien kommt es in den meisten Referenznetzgebieten immer stärker dazu, dass in einem Großteil der Zeitschritte lastseitige Abregelung nötig ist.
Hierdurch nimmt das relative Senkungspotential des lastseitigen Abregelungsbedarfs ab, da der Abregelungsbedarf verstärkt verschoben, aber nicht verhindert wird.
Die Untersuchung der \SzeFirmenparkplatz ergibt, dass beim Referenz-Laden in den meisten Fällen der Bedarf an last- und erzeugerseitiger Abregelung im Vergleich zum Antriebswende-Szenario geringer ausfällt.
Hierfür verantwortlich ist in der Regel der hohe Anteil an \glspl{PVA} in den \gls{NS}-Netzen mit einem hohen Anteil an Ladevorgängen des \UC \zHdot.
Sind diese \gls{NS}-Netze jedoch bereits stark ausgelastet, dann erhöht sich der lastseitige Abregelungsbedarf in der \SzeFirmenparkplatz gegenüber dem Antriebswende-Szenario.\medskip

Zusammenfassend bietet nur das aktive Residuallast-Laden die Möglichkeit, die Integration von \gls{FEE} Anlagen in die Verteilnetze zu erleichtern.
Der Erfolg des Residuallast-Ladens hängt hierbei jedoch von vielen Randbedingungen ab.
Lastseitig weist häufig das präventive reduzierte Laden die besten Ergebnisse auf und bietet eine gute alternative zu dem aktiv gesteuerten Residuallast-Laden.
Die Ladegruppen sind hingegen nicht dafür geeignet die Netzintegration von \glspl{EPKW} zu erleichtern.\medskip

Um zu überprüfen, ob das Residuallast-Laden den last- und erzeuerseitigen Abregelungsbedarf noch weiter senken kann, sollten zwei Optionen untersucht werden.
Hierbei sollte jedoch darauf geachtet werden, dass diese Konzepte auch in der Realität abgebildet werden können.

\begin{enumerate}
	\item Zum einen können dem Residuallast-Laden Randbedingungen hinzugefügt werden.
	So kann beispielsweise die Belastungsgrenze des jeweiligen \gls{MS}-\gls{NS}-\glspl{USW} beachtet werden oder/und die Wirkleistung der installierten \gls{FEE} Erzeugerkapazitäten in den \gls{NS}-Netzen als ein Gewichtungsfaktor in die Zuteilung der Ladezeitfenster eingehen.
	\item Zum anderen kann statt der globalen Residuallast innerhalb des Netzgebietes auch die Residuallast einzelner Abschnitte des Netzgebietes als Leitlinie dienen.
	Hierbei kann beispielsweise die Residuallast der einzelnen \gls{MS}-Abgänge genutzt werden oder sogar die Residuallast einzelner \gls{NS}-Netze.
\end{enumerate}


% Je höher der Einfluss des Ladebedarfs auf die Residuallast ausfällt, desto mehr Zeitschritte werden für das Laden der Fahrzeuge verwendet.
% je mehr MS Ebene, desto besser wird Referenz-Laden ggü den Ladegruppen