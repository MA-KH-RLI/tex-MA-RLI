\section{Schlussbetrachtung und Ausblick}\label{chap:schlussbetrachtung}

Das Ziel dieser Arbeit war es, die Auswirkungen verschiedener präventiver und aktiver Ladestrategien bei unterschiedlichen Durchdringungstiefen von \glspl{EPKW} auf die drei Netzgebietsklassen \gls{PV}-, Wind- und Last-dominiert aufzuzeigen.
Dafür wurde der jeweils nötige last- und erzeugerseitige Abregelungsbedarf innerhalb der untersuchten \gls{MS}-Netze inklusive darunterliegender \gls{NS}-Netze quantifiziert.\medskip

Mit \glsi{SIMBEV} wurde ein Software Tool mitentwicklet, mit dessen Hilfe plausible Fahrtprofile von \gls{EPKW} erzeugt werden können.
Die Fahrtprofile zeigen, dass ein Großteil der Ladevorgänge im privaten Bereich stattfindet.
Bei privaten Ladevorgängen wird zudem ersichtlich, dass die Ladung der \glspl{EPKW} nur einen geringen Anteil der Standzeit ausmacht.
In den untersuchten Szenarien entstehen die stärksten Leistungsspitzen am Morgen durch das Laden am Arbeitsplatz, während das Laden zu Hause den größten Energiebedarf aufweist.
Die anschließende Verortung des Ladebedarfs führt teilweise zu unplausiblen Belastungen einzelner \gls{NS}-Netze innerhalb der Referenznetzgebiete.
In der Realität würde in solchen Situationen vermutlich ein umfänglicher Netzausbau oder sogar ein Netzneubau vorgenommen, welches innerhalb dieser Arbeit nicht abgebildet werden kann.\medskip

Mit einem steigenden Ladebedarf, aufgrund eines erhöhten Hochlaufs der \glspl{EPKW} in den Szenarien nimmt der Bedarf an lastseitigen Abregelungen in den meisten Referenznetzgebieten immer stärker zu.
Hierdurch verringert sich das relative Senkungspotential des lastseitigen Abregelungsbedarfs, da der Abregelungsbedarf verstärkt verschoben, jedoch nicht verhindert werden kann.
Die Untersuchung der \SzeFirmenparkplatz ergibt, dass beim Referenz-Laden in den meisten Fällen der Bedarf an last- und erzeugerseitiger Abregelung im Vergleich zum Antriebswende-Szenario geringer ausfällt.
Hierfür verantwortlich ist in der Regel der hohe Anteil an \glspl{PVA} in den \gls{NS}-Netzen mit einem hohen Anteil an Ladevorgängen des \UC \zHdot.
Sind diese \gls{NS}-Netze lastseitig bereits stark ausgelastet, dann erhöht sich der lastseitige Abregelungsbedarf in der \SzeFirmenparkplatz gegenüber dem Antriebswende-Szenario.\medskip

Bei den Ladestrategien führen die präventiven Ladestrategien zu einer Erhöhung der Spreizung zwischen der maximalen und minimalen Residuallast.
Dies kann als Indikator für eine möglicherweise benötigte Erhöhung der Dimensionierung der Netze gegenüber dem Referenz-Laden gedeutet werden.
Der negative Effekt der präventiven Ladestrategien lässt sich primär in Netzen mit einem hohen Anteil an \glspl{PVA} beobachten.
Am frühen Nachmittag ist in diesen Netzen aufgrund der präventiven Ladestrategien ein geringerer Ladebedarf zu verzeichnen.
Da in diesem Zeitfenster die Einspeisung der \glspl{PVA} in der Regel hoch ausfällt, wird die Residuallast gesenkt.
Von diesem Effekt sind in erster Linie Ladevorgänge des \UC \zH betroffen.
Da in \gls{NS}-Netzen mit einem hohen Anteil an Ladevorgängen des \UC \zH in der Regel auch viele \glspl{PVA} verortet sind, erhöht sich in diesen Fällen aufgrund der präventiven Ladestrategien der Abregelungsbedarf von \gls{FEE}.
Dies gilt vor allem für die \gls{PV}-dominierten Netze.
Aufgrund der in den Netzen installierten \gls{PV}-Kapazitäten sind jedoch auch die Wind- und Last-dominierten Netze hiervon betroffen.
In den Wind-dominierten Netzen fällt der Effekt aufgrund des niedrigeren Anteils von \glspl{PVA} an den Erzeugerkapazitäten relativ geringer aus.\medskip

Im Gegensatz zu den präventiven Ladestrategien kann durch das Residuallast-Laden sowohl der maximale Last- als auch Einspeisefall im Netzgebiet gesenkt werden.
Erweiternd kann der Abregelungsbedarf von \gls{FEE} in den \gls{PV}- und Last-dominierten Netzen gesenkt werden.
Hierbei werden viele Ladevorgänge in die Mittagszeit verschoben und es findet ein besserer Ausgleich zwischen Last und Erzeugung statt.
In den Wind-dominierten Netzen zeigt sich demgegenüber ein erhöhter Abregelungsbedarf von \gls{FEE}, wobei sich zwei Effekte beobachten lassen.
Zum einen befindet sich in den Wind-dominierten Netzen zusätzlich ein gewisser Anteil an \glspl{PVA}, welcher wie zuvor beschrieben vermehrt in \gls{NS}-Netzen mit einem hohen Anteil an Ladevorgängen des \UC \zH verortet ist.
Aufgrund einer niedrigen Residuallast in der Nacht werden viele Ladevorgänge des \UC \zH in die Nacht verschoben.
Zum anderen befinden sich Wind-Kapazitäten nicht in örtlicher Nähe zu der privaten Ladeinfrastruktur und werden auf der \gls{MS}-Ebene angeschlossen.
Hierdurch wird der Abregelungsbedarf der Wind-Kapazitäten durch die Ladestrategien kaum beeinflusst, während sich der Abregelungsbedarf der \glspl{PVA}, und somit in Summe der Abregelungsbedarf von \gls{FEE}, erhöht.\medskip

Der lastseitige Abregelungsbedarf kann in den meisten Fällen sowohl durch das reduzierte Laden als auch das Residuallast-Laden gesenkt werden.
Dabei ist insbesondere der Erfolg des Residuallast-Ladens von mehreren Randbedingungen abhängig.
Zum einen ist es aufgrund von Ausgleichseffekten zwischen der Last und Erzeugung entscheidend, ob sich die Erzeugerkapazitäten in den gleichen \gls{NS}-Netzen befinden, in denen auch die Ladevorgänge stattfinden, und ob Ladevorgänge und Einspeisung innerhalb eines \gls{NS}-Netzes gleichzeitig stattfinden.
Zum anderen ist von großer Bedeutung, wie stark die einzelnen \gls{NS}-Netze ausgelastet sind.
Wenn ein \gls{NS}-Netz bereits innerhalb des Großteils der Zeitpunkte überlastet ist, kann der lastseitige Abregelungsbedarf in der Regel nur noch verschoben, jedoch nicht verhindert werden.
Sind diese Bedingungen gegeben, kann durch das Residuallast-Laden ein hoher Anteil des lastseitigen Abregelungsbedarfs vermieden werden.
Ist dies hingegen nicht der Fall, führen hohe Gleichzeitigkeiten, welche durch das Residuallast-Laden entstehen können, zu negativen lastseitigen Auswirkungen.
So entsteht primär in den Wind-dominierten Netzen der Effekt, dass sich das Residuallast-Laden an einer globalen Residuallast im gesamten Netzgebiet orientiert, welche nur bedingt die lokale Situation in den einzelnen \gls{NS}-Netzen widerspiegelt.
In diesen Fällen erweist sich das reduzierte Laden als besser geeignet, um den lastseitigen Abregelungsbedarf zu senken.
Die Ladegruppen führen hingegen nur zu einer geringen Senkung dieses gegenüber dem Referenz-Laden.
So kann die Gleichzeitigkeit durch die Ladegruppen nur in einem geringen Maße beeinflusst werden.
Auf der \gls{MS}-Ebene erhöht sich zudem die maximale Last gegenüber dem Referenz-Laden, da reduzierte Ladevorgänge durch Ladevorgänge in Ladegruppen ersetzt werden.\medskip

Zusammenfassend zeigt sich, dass sowohl präventive als auch aktive Ladestrategien die Netzintegration von \gls{EPKW} unterstützen können.
Das präventive reduzierte Laden erweist sich als besonders erfolgreich den lastseitigen Abregelungsbedarf zu mindern.
Demgegenüber können durch die ebenfalls präventiven Ladegruppen keine signifikanten Reduzierungen des Abregelungsbedarfs erreicht werden.
Unter den richtigen Bedingungen kann der lastseitige Abregelungsbedarf durch das aktive Residuallast-Laden am stärksten gesenkt werden.
Erweiternd kann das Residuallast-Laden auch die Netzintegration von \gls{FEE} unterstützen.
Dies können die präventiven Ladestrategien nicht leisten und führen sogar zu einer Verschlechterung des Abregelungsbedarfs von \gls{FEE}.
Der Erfolg des Residuallast-Ladens ist davon abhängig, ob die globale Residuallast im \gls{MS}-Netzgebiet die Situationen in den einzelnen \gls{NS}-Netzen näherungsweise gut widerspiegelt.
Ist dies nicht gegeben, erweist sich das reduzierte Laden als die beste Alternative, um den lastseitigen Abregelungsbedarf zu minimieren und somit die Netzintegration von \glspl{EPKW} zu unterstützen.\medskip

Neben den untersuchten Ladestrategien, gibt es noch eine Vielzahl von denkbaren Ladestrategien, welche ebenfalls auf ihre netzdienlichen Effekte untersucht werden sollten.
Erweiternd zu den netzdienlichen Ladestrategien sollte auch bei der Verortung der Ladeinfrastruktur auf ein möglichst netzdienliches Vorgehen geachtet werden, indem Gegebenheiten des Netzes miteinbezogen werden.
Weiterhin sollte \glsi{SIMBEV} dahingehend erweitert werden, einen Zeitraum von mindestens einem Jahr und saisonale Schwankungen im Verbrauch der \gls{EPKW} und dem Fahrtverhalten der Fahrzeugnutzer\(^*\)innen abbilden zu können.
Hierdurch würde die Realitätsnähe des Tools erhöht und eine größere Anzahl von unterschiedlichen Bedarfsfällen abgedeckt werden.\medskip

Konkret sollte geprüft werden, ob das Residuallast-Laden erweitert und dadurch verbessert werden kann.
Hierbei sollten zwei Optionen untersucht werden, wobei darauf zu achten ist, dass diese Konzepte auch in der Realität abgebildet werden können.
Zum einen können dem Residuallast-Laden Randbedingungen hinzugefügt werden.
So kann beispielsweise die Belastungsgrenze der jeweiligen \gls{ONS} beachtet werden und/oder die Wirkleistung der installierten \gls{FEE} Erzeugerkapazitäten in den \gls{NS}-Netzen als ein Gewichtungsfaktor in die Ladestrategie mit eingehen.
Zum anderen kann statt der globalen Residuallast innerhalb des Netzgebietes auch die Residuallast einzelner Abschnitte des Netzgebietes als Leitlinie dienen.
Hierbei kann beispielsweise die Residuallast der einzelnen \gls{MS}-Abgänge oder sogar die Residuallast einzelner \gls{NS}-Netze genutzt werden.
Es ist zu vermuten, dass aufgrund dieser Maßnahmen lokale Engpässe in den Netzen besser berücksichtigt werden und die Netzintegration von \gls{EPKW} stärker unterstützt werden kann.

% je mehr MS Ebene, desto besser wird Referenz-Laden ggü den Ladegruppen