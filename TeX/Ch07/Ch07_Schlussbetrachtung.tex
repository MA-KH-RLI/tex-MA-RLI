\section{Schlussbetrachtung und Ausblick}\label{chap:schlussbetrachtung}

Mit \glsi{SIMBEV} wurde ein Software Tool mitentwicklet, mit dessen Hilfe plausible Fahrtprofile von \gls{EPKW} erzeugt werden können.
Die Fahrtprofile zeigen, dass ein Großteil der Ladevorgänge im privaten Bereich stattfindet.
Bei privaten Ladevorgängen wird zudem ersichtlich, dass die Ladung der \glspl{EPKW} nur einen geringen Anteil der Standzeit ausmacht.
In den untersuchten Szenarien entstehen die stärksten Leistungsspitzen am Morgen durch die hohe Gleichzeitigkeit des Wegezwecks \Arbeitdot.
Den größten Energiebedarf weist hingegen der \UC \zH auf.
Die anschließende Verortung des Ladebedarfs führt teilweise zu unplausiblen Belastungen einzelner \gls{NS}-Netze innerhalb der Referenznetzgebiete.
In der Realität würde in solchen Situationen vermutlich ein umfangreicher Netzausbau oder sogar ein Netzneubau vorgenommen, welches innerhalb dieser Arbeit nicht abgebildet werden kann.\medskip

Mit einem steigenden Ladebedarf, aufgrund eines erhöhten Hochlaufs der \glspl{EPKW} in den Szenarien, nimmt der Bedarf an lastseitigen Abregelungen in den Referenznetzgebieten immer stärker zu und immer mehr Zeitpunkte sind von Abregelungen betroffen.
Hierdurch verringert sich das relative Senkungspotential der Ladestrategien auf den lastseitigen Abregelungsbedarf, da dieser verstärkt nur noch verschoben, jedoch nicht verhindert wird.

Die Untersuchung der \SzeFirmenparkplatz ergibt, dass beim Referenz-Laden in den meisten Fällen der Bedarf an last- und erzeugerseitiger Abregelung im Vergleich zum Antriebswende-Szenario geringer ausfällt.
Hierfür verantwortlich ist der hohe Anteil an \glspl{PVA} in den \gls{NS}-Netzen mit einem hohen Anteil an Ladevorgängen des \UC \zHdot.
In einzelnen Fällen sind diese \gls{NS}-Netze lastseitig jedoch so stark ausgelastet, dass sich der lastseitige Abregelungsbedarf in der \SzeFirmenparkplatz gegenüber dem Antriebswende-Szenario erhöht.\medskip

Bei den Ladestrategien führen die präventiven Ladestrategien zu einer Erhöhung der Spreizung zwischen der maximalen und minimalen Residuallast.
Dies kann als Indikator für eine möglicherweise benötigte Erhöhung der Dimensionierung der Netze gegenüber dem Referenz-Laden gedeutet werden.
Der negative Effekt der präventiven Ladestrategien lässt sich primär in Netzen mit einem hohen Anteil an \glspl{PVA} beobachten.
Am frühen Nachmittag ist in diesen Netzen aufgrund der präventiven Ladestrategien ein geringerer Ladebedarf des \UC \zH zu verzeichnen.
Da in diesem Zeitfenster die Einspeisung der \glspl{PVA} in der Regel hoch ausfällt, wird die Residuallast gesenkt.
In \gls{NS}-Netzen mit einem hohen Anteil an Ladevorgängen des \UC \zH sind in der Regel auch viele \glspl{PVA} verortet, weshalb sich der Abregelungsbedarf von \gls{FEE} erhöht.
Aufgrund der in den Wind- und Last-dominierten Netzen installierten \gls{PV}-Kapazitäten sind jedoch auch diese hiervon betroffen.
In den Wind-dominierten Netzen fällt der Effekt aufgrund des niedrigeren Anteils von \glspl{PVA} an den Erzeugerkapazitäten relativ betrachtet geringer aus.\medskip

Im Gegensatz zu den präventiven Ladestrategien wird durch das Residuallast-Laden sowohl der maximale Last- als auch Einspeisefall im Netzgebiet gemindert.
Erweiternd wird der Abregelungsbedarf von \gls{FEE} in den \gls{PV}- und Last-dominierten Netzen abgesenkt.
Hierbei werden viele Ladevorgänge in die Mittagszeit verschoben und es findet ein besserer Ausgleich zwischen Last und Erzeugung statt.
In den Wind-dominierten Netzen zeigt sich demgegenüber ein erhöhter Abregelungsbedarf von \gls{FEE}, wobei sich zwei Effekte beobachten lassen.
Zum einen befindet sich in den Wind-dominierten Netzen zusätzlich ein gewisser Anteil an \glspl{PVA}, welche wie zuvor beschrieben vermehrt in \gls{NS}-Netzen mit einem hohen Anteil an Ladevorgängen des \UC \zH verortet sind.
Aufgrund einer niedrigen Residuallast in der Nacht werden jedoch viele Ladevorgänge des \UC \zH in die Nacht verschoben.
Zum anderen befinden sich Windkraftanlagen nicht in örtlicher Nähe zu der Ladeinfrastruktur und werden auf der \gls{MS}-Ebene angeschlossen.
So wird auf der einen Seite der Abregelungsbedarf der Windkraftanlagen durch die Ladestrategien kaum beeinflusst, während sich auf der anderen Seite der Abregelungsbedarf der \glspl{PVA}, und somit in Summe der Abregelungsbedarf der \gls{FEE}, erhöht.\medskip

Der lastseitige Abregelungsbedarf wird in den meisten Fällen sowohl durch das reduzierte Laden als auch das Residuallast-Laden gesenkt.
Dabei ist der Erfolg des Residuallast-Ladens von mehreren Randbedingungen abhängig.
Zum einen ist es aufgrund von Ausgleichseffekten zwischen der Last und Erzeugung entscheidend, ob sich die Erzeugerkapazitäten in den gleichen \gls{NS}-Netzen befinden, in denen auch die Ladevorgänge stattfinden, und ob Ladevorgänge und Einspeisung innerhalb eines \gls{NS}-Netzes gleichzeitig stattfinden.
Zum anderen ist von großer Bedeutung, wie stark die einzelnen \gls{NS}-Netze ausgelastet sind.
Sind diese Bedingungen gegeben, wird durch das Residuallast-Laden ein hoher Anteil des lastseitigen Abregelungsbedarfs vermieden.
Ist dies hingegen nicht der Fall, führen hohe Gleichzeitigkeiten, welche durch das Residuallast-Laden entstehen können, zu negativen lastseitigen Auswirkungen.
So entsteht primär in den Wind-dominierten Netzen der Effekt, dass sich das Residuallast-Laden an einer globalen Residuallast im gesamten Netzgebiet orientiert, welche nur schlecht die lokale Situation in den einzelnen \gls{NS}-Netzen widerspiegelt.
In diesen Fällen erweist sich das reduzierte Laden als besser geeignet, um den lastseitigen Abregelungsbedarf zu senken.

Die Ladegruppen führen hingegen nur zu einer geringen Senkung des Abregelungsbedarfs gegenüber dem Referenz-Laden, da die Gleichzeitigkeit durch die Ladegruppen nur in einem geringen Maße beeinflusst wird.
Auf der \gls{MS}-Ebene erhöht sich zudem die maximale Last gegenüber dem Referenz-Laden, da reduzierte Ladevorgänge durch Ladevorgänge in Ladegruppen ersetzt werden.\medskip

Zusammenfassend zeigt sich, dass sowohl präventive als auch aktive Ladestrategien die Netzintegration von \gls{EPKW} unterstützen können.
Dabei erweist sich das präventive reduzierte Laden als besonders erfolgreich den lastseitigen Abregelungsbedarf zu mindern.
Demgegenüber wird durch die ebenfalls präventiven Ladegruppen keine signifikante Reduzierungen des Abregelungsbedarfs erreicht.
Unter den richtigen Bedingungen wird durch das aktive Residuallast-Laden der lastseitige Abregelungsbedarf am stärksten gesenkt und die Netzintegration von \gls{FEE} unterstütz.
Dies können die präventiven Ladestrategien nicht leisten und führen sogar zu einer Erhöhung des Abregelungsbedarfs von \gls{FEE}.
Der Erfolg des Residuallast-Ladens ist allerdings stark davon abhängig, ob die globale Residuallast im \gls{MS}-Netzgebiet die Situationen in den einzelnen \gls{NS}-Netzen näherungsweise gut widerspiegelt.
Ist dies nicht gegeben, erweist sich das reduzierte Laden als die beste Alternative, um den lastseitigen Abregelungsbedarf zu minimieren und somit die Netzintegration von \glspl{EPKW} zu unterstützen.\bigskip

Neben den untersuchten Ladestrategien, existiert eine Vielzahl von denkbaren Ladestrategien, welche ebenfalls auf ihre netzdienlichen Effekte untersucht werden sollten.
Erweiternd zu den netzdienlichen Ladestrategien sollte auch bei der Verortung der Ladeinfrastruktur auf ein möglichst netzdienliches Vorgehen geachtet werden, indem Gegebenheiten des Netzes miteinbezogen werden.
Weiterhin sollte \glsi{SIMBEV} dahingehend erweitert werden, einen Zeitraum von mindestens einem Jahr und saisonale Schwankungen im Verbrauch der \gls{EPKW} und dem Verkehrsverhalten der Fahrzeugnutzer\(^*\)innen abbilden zu können.
Hierdurch würde die Realitätsnähe des Tools erhöht und eine größere Anzahl von unterschiedlichen Bedarfsfällen abgedeckt werden.\medskip

Konkret sollten zwei Optionen geprüft werden, um das Residuallast-Laden zu erweitern.
Zum einen können dem Residuallast-Laden Randbedingungen hinzugefügt werden.
Es empfiehlt sich die Belastungsgrenze der jeweiligen \gls{ONS} zu beachten und/oder die Wirkleistung der installierten \gls{FEE} Erzeugerkapazitäten in den \gls{NS}-Netzen als ein Gewichtungsfaktor in die Ladestrategie mit aufzunehmen.
Zum anderen könnte statt der globalen Residuallast innerhalb des Netzgebietes auch die Residuallast einzelner Abschnitte des Netzgebietes als Leitlinie dienen.
Hierbei könnte beispielsweise die Residuallast der einzelnen \gls{MS}-Abgänge oder sogar der einzelnen \gls{NS}-Netze genutzt werden.
Es ist zu vermuten, dass aufgrund dieser Maßnahmen lokale Engpässe in den Netzen besser berücksichtigt und die Netzintegration von \gls{EPKW} stärker unterstützt werden.

% je mehr MS Ebene, desto besser wird Referenz-Laden ggü den Ladegruppen