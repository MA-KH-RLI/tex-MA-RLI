
\section{Szenariorahmen}\label{chap:Szenariorahmen}

Innerhalb des Workshops \glqq Neue Verbraucher und elektrische Flexbilitäten\grqq{} wurden Fragestellungen diskutiert, die sich mit der Integration neuer Verbraucher und der Nutzung elektrischer Flexibilitäten in Übertragungs- und Verteilnetzen befassen \cite{RLI2020}.
Hierbei wurden unter anderem die Ergebnisse der Metaanalyse (s. \autoref{chap:Literatur}) genutzt, um mit Branchenexpert$^*$innen Forschungsfragen zu entwerfen und zukünftige Entwicklungen zu diskutieren.
Die wichtigsten Ergebnisse des Workshops werden kurz vorgestellt und dienen als Leitlinien für die Erstellung eines geeigneten Szenariorahmens für diese Arbeit innerhalb dieses Kapitels.\medskip

Insgesamt werden drei Szenarien und eine Szenarette untersucht.
Die Szenarien unterscheiden sich ausschließlich im Hochlauf der \glspl{EPKW}.
Bei der Szenarette wird hingegen eine Veränderung in der Verfügbarkeit der Ladeinfrastruktur untersucht.
Auf diese Weise sollen getrennt die Auswirkungen der beiden Parameter aufgezeigt werden.


\subsection{Ergebnisse des Workshops {--} Neue Verbraucher und elektrische Flexbilitäten}

In diesem Kapitel werden die für diese Arbeit wichtigsten Ergebnisse des Workshops \glqq Neue Verbraucher und elektrische Flexbilitäten\grqq{} aufgelistet.
Zu den Teilnehmern$^*$innen des Workshops zählten mehrheitlich Mitarbeiter$^*$innen von Netzbetreibern, Forschungsinstituten und energiewirtschaftlichen Unternehmen.
Die folgende Auflistung entspricht den relevantesten Ergebnissen mehrerer Diskussionen und Umfragen innerhalb des Workshops.

\begin{itemize}
%	\item Die Untersuchung eines Mobilitätswende-Szenarios ist von besonders hohem Interesse
	\item Private Ladeinfrastruktur zu Hause oder auf Firmenparkplätzen besitzt die größte Attraktivität für Verbraucher.
%	\item Es wird davon ausgegangen, dass \num{2050} etwa \num{15} \glspl{EPKW} auf einen öffentlichen Ladepunkt kommen
	\item Eine regionale Konzentration von \glspl{EPKW} wird nur während der Markthochlaufphase eine relevante Rolle spielen.
	\item Marktorientierte Ladestrategien können Gleichzeitigkeiten und damit den Netzausbaubedarf erhöhen.
	\item Die großtechnische Umsetzung von \gls{V2G}-Anwendungen wird als unwahrscheinlich eingestuft.
	
\end{itemize}


\subsection{Szenarien}

In diesem Kapitel erfolgt eine Beschreibung der Annahmen zu den untersuchten Szenarien und somit den Eingangsparametern für die Simulation des Einflusses der Netzintegration von \glspl{EPKW}.
Hierzu zählen der Ausbau regenerativer Energien, der konventionelle Stromverbrauch, der Hochlauf von \glspl{WP} und \glspl{EPKW} sowie deren technische Parameter.
Der Fokus liegt hierbei auf der Aufstellung von Annahmen zum Hochlauf von \glspl{EPKW} und zur Verfügbarkeit der Ladeinfrastruktur.
Insgesamt werden drei Szenarien und eine daraus abgeleitete Sensitivität in Form einer Szenarette untersucht.


\subsubsection{Ausbau regenerativer Energien}

Die Annahmen zum Hochlauf der regenerativen Erzeugerkapazitäten werden innerhalb der Szenarien als konstant angenommen.
Hierdurch werden Vermischungseffekte aufgrund der Variation mehrerer Eingangsparameter vermieden und der Einfluss der Elektromobilität kann getrennt betrachtet werden.
Die Annahmen zum Hochlauf der Erzeugerkapazitäten werden aus dem Szenario \ego des \glsi{OPENEGO} Projekts entnommen und sind in \autoref{tab:EE-RampUp} dargestellt.

Auf Grundlage des Status Quo im Jahr \num{2015} werden die Erzeugerkapazitäten entsprechend der aktuellen Verteilung gewichtet regionalisiert.
Hierbei werden maximale Ausbaupotentiale berücksichtigt, indem beispielsweise Weißflächen für den Ausbau von Onshore Windkraftanlagen berücksichtigt werden.
Die Regionalisierung der Erzeugerkapazitäten erfolgte im Rahmen des \glsi{OPENEGO} Projektes und eine genaue Beschreibung der Methodik ist im Projektabschlussbericht \cite{Mueller2019} festgehalten.


{
\renewcommand{\arraystretch}{1.2}% grßerer Zeilenabstand
\sisetup{range-phrase=~{--}~}% Gedankenstrich statt "bis" bei SIrange
\begin{table}[H]
	\begin{center}
		\caption{Hochlaufzahlen der regenerativen Erzeugerkapazitäten}
		\begin{tabu} to 0.45\textwidth {X[1] X[1.5, r]}
			\toprule
			              & Erzeugerkapazitäten \\ \midrule
			Wind Onshore  & \SI{98.4}{\gw}                      \\
			Wind Offshore & \SI{27.0}{\gw}                      \\
			Photovoltaik  & \SI{97.8}{\gw}                      \\
			Biomasse      & \SI{27.8}{\gw}                      \\
			Wasserkraft   & \SI{3.2}{\gw}                       \\ \bottomrule
		\end{tabu}
		\label{tab:EE-RampUp}
	\end{center}
	\vspace{-3mm}%Put here to reduce too much white space after your table
\end{table}
}

Die Zeitreihen für die Erzeugung aus Wind- und \gls{PV}-Anlagen werden der \glsi{OEP} \cite{OEP} entnommen und sind frei verfügbar.
Im Falle der Energieerzeugung aus Biomasse und Wasserkraft wird eine konstante Erzeugung angenommen.
So waren in Deutschland im Jahr \num{2019} \SI{9983}{\mw} an Biomasse- und \SI{5595}{\mw} an Wasserkraftwerken installiert, die insgesamt \SI{50009}{\gwh} bzw. \SI{20058}{\gwh} Energie erzeugten \cite{BMWi2020}.
Der hieraus resultierende Leistungsfaktor von \num{0.57} bzw. \num{0.41} wird auf alle Biomasse- und Wasserkraft-Erzeugerkapazitäten unverändert umgelegt.
In den untersuchten \gls{MS}-Netzgebieten finden sich keine konventionellen Kraftwerke.


\subsubsection{Konventioneller Stromverbrauch}

Bei dem konventionellen Stromverbrauch handelt es sich um geo­gra­fisch hochaufgelöste Zeitreihen, die jeweils einem sogenannten \Lastgebiet zugeordnet werden.
Der konventionelle Stromverbrauch entstammt parallel zum Ausbau der regenerativen Energien aus dem \glsi{OPENEGO} Projekt und wurde im entsprechenden Projektbericht \cite{Mueller2019} ausführlich beschrieben.
An dieser Stelle soll die Methodik kurz zusammengefasst werden.\medskip

Die \Lastgebiete entsprechen einer geo­gra­fischen Einheit, welcher ein elektrischer Verbrauch zugeordnet wird.
Um die Datenmengen möglichst gering zu halten, werden innerhalb eines \Lastgebietes räumlich nahe liegende Landnutzungsflächen zusammengefasst.
Zusätzlich wird den \Lastgebieten eine Bevölkerungsanzahl und der Anteil an den Gesamtlandnutzungsflächen der Sektoren Gewerbe, Handel, Dienstleistungen, Industrie, Wohnen und Landwirtschaft zugeteilt.
Anhand dieser Charakteristika erfolgt abschließend eine anteilige Zuordnung des gesamten Stromverbrauches und der Zeitreihen der Sektoren auf die \Lastgebietedot.


\subsubsection{Wärmepumpen}

Der Hochlauf an \glspl{WP} in Deutschland wird innerhalb der Szenarien als feste Eingangsgröße angenommen.
Die Hochlaufzahlen und der Jahresverbrauch entsprechen dem Szenario C~\num{2035} des \glspl{NEP} \numrange[range-phrase=~{--}~]{2021}{2035} \cite{BNetzA2020} und sind in \autoref{tab:WP-RampUp} dargestellt.\medskip

Der Jahresstromverbrauch der \glspl{WP} wird anteilig anhand des Stromverbrauches des Haushaltssektors eines Netzgebietes bezogen auf den Gesamtstromverbrauch des Haushaltssektors im Szenario \ego auf die einzelnen Netzgebiete verteilt.
Innerhalb eines Netzgebietes erfolgt die Regionalisierung gewichtet anhand des Stromverbrauches der einzelnen Haushaltslasten.
Der zeitliche Verlauf der Last wird der \textit{E-Mobility Study} \cite{Schachler} entnommen und auf alle \glspl{WP} umgelegt.

{
\renewcommand{\arraystretch}{1.2}% grßerer Zeilenabstand
\sisetup{range-phrase=~{--}~}% Gedankenstrich statt "bis" bei SIrange
\begin{table}[H]
	\begin{center}
		\caption{Hochlaufzahlen für Wärmepumpen}
		\begin{tabu} to 0.6\textwidth {X[1.5] X[1, r]}
			\hline
										 & Wärmepumpen \\ \hline
			Anzahl in \si{\MioStkSC}     & \num{7.0}   \\
			Leistung in \si{\gw}         & \num{21.0}  \\
			Jahresverbrauch in \si{\twh} & \num{22.4}  \\ \hline
		\end{tabu}
		\label{tab:WP-RampUp}
	\end{center}
	\vspace{-3mm}%Put here to reduce too much white space after your table
\end{table}
}


\subsubsection{Elektromobilität}\label{chap:EMob_Szenarien}

Bei der Elektromobilität müssen neben den Annahmen zum Fahrzeughochlauf auch die technischen Daten der \gls{EPKW} und die Verfügbarkeit der Ladeinfrastruktur berücksichtigt werden.


\paragraph{Fahrzeughochlauf:}
Der Fahrzeughochlauf unterscheidet sich je nach Szenario.
Insgesamt spiegeln alle Szenarien unterschiedlich starke Durchdringungen des \glspl{MIV} mit \glspl{EPKW} wider.

{
\renewcommand{\arraystretch}{1.2}% grßerer Zeilenabstand
\sisetup{range-phrase=~{--}~}% Gedankenstrich statt "bis" bei SIrange
\begin{table}[H]
	\begin{center}
		\caption{E-Pkw Hochlaufzahlen je Szenario}
		\begin{tabu} to 0.6\textwidth {X[1] X[2, r]}
			\hline
			Szenario         & E-PKW in \si{\MioStk}	\\ \hline
			NEP C~\num{2035} & \num{14.0}               \\
			Referenz         & \num{25.1}               \\
			Mobilitätswende  & \num{37.7}               \\
			Antriebswende    & \num{47.7}               \\ \hline
		\end{tabu}
		\label{tab:SzenarienRampUp}
	\end{center}
	\vspace{-3mm}%Put here to reduce too much white space after your table
\end{table}
}

In \autoref{tab:SzenarienRampUp} sind die Annahmen der drei Szenarien zum Hochlauf von \glspl{EPKW} dargestellt.
Das erste Szenario entspricht mit einem Fahrzeughochlauf von \SI{14}{\Mio} \gls{EPKW} den Annahmen des Szenarios C~\num{2035} des \gls{NEP} \numrange[range-phrase=~{--}~]{2021}{2035} \cite{BNetzA2020}.
Die Hochlaufzahlen für das Referenzszenario spiegeln mit \SI{25.1}{\Mio} \gls{EPKW} den summierten Median der Literaturrecherche für den Fahrzeughochlauf an \glspl{BEV} und \glspl{PHEV} im Jahr \num{2050} wider.
Das Antriebswende-Szenario geht hingegen von einer vollständigen Elektrifizierung des Fahrzeugbestandes von \SI{47.7}{\Mio} \gls{EPKW} vom \DTMdate{2020-01-01} \cite{KBA2020} aus.

{
\renewcommand{\arraystretch}{1.2}% grßerer Zeilenabstand
\sisetup{range-phrase=~{--}~}% Gedankenstrich statt "bis" bei SIrange
\begin{table}[H]
	\begin{center}
		\caption{Aufteilung der E-Pkw auf die einzelnen Fahrzeugtypen und -klassen}
		\begin{tabu} to 0.4\textwidth {X[1.5] X[1, r]}
			\toprule
			Fahrzeugklasse    & Anteil  \\ \midrule
			BEV Kleinwagen    & \SI{15.9}{\percent}               \\
			BEV Mittelklasse  & \SI{35.3}{\percent}               \\
			BEV Oberklasse    & \SI{10.5}{\percent}               \\
			PHEV Kleinwagen   & \SI{9.8}{\percent}                \\
			PHEV Mittelklasse & \SI{21.9}{\percent}               \\
			PHEV Oberklasse   & \SI{6.5}{\percent}                \\ \bottomrule
		\end{tabu}
		\label{tab:CarSplit}
	\end{center}
	\vspace{-3mm}%Put here to reduce too much white space after your table
\end{table}
}

Die Aufteilung des \gls{EPKW}-Bestands in \glspl{BEV} und \glspl{PHEV} erfolgt anhand des ermittelten Medians der Literaturrecherche für den Fahrzeughochlauf an \glspl{BEV} und \glspl{PHEV} für das Stützjahr \num{2050}.
Innerhalb der \gls{EPKW}-Typen wird erweiternd eine Einteilung in die Fahrzeugklassen Kleinwagen, Mittelklasse und Oberklasse vorgenommen.
Es wird davon ausgegangen, dass sich die Einteilung in Fahrzeugklassen zwischen \glspl{BEV} und \glspl{PHEV} nicht unterscheidet.
So gibt es in den Szenarien zwar insgesamt mehr \glspl{BEV} als \glspl{PHEV}, aber das Verhältnis zwischen den Fahrzeugklassen ist zwischen den beiden \gls{EPKW}-Typen konstant.
Die Einteilung in \autoref{tab:CarSplit} entspricht der Statistik des Kraftfahrt-Bundesamtes \cite{KBASegments2020} vom \DTMdate{2020-01-01}.
Es wird somit von einer Fortschreibung des aktuellen Verbraucherverhaltens ausgegangen.
Eine genaue Zuordnung der \gls{PKW}-Segmente in die Klassen wird im Anhang in \autoref{tab:KBASegments} und \autoref{tab:Segments} dargestellt.


\paragraph{Technische Daten:}

Die technischen Daten der \gls{EPKW} sind klassenspezifisch und werden innerhalb ihrer Klasse als homogen angenommen.
In \autoref{tab:TechPowerCap} ist die fahrzeugseitige maximale Ladeleistung und die nutzbare Batteriekapazität der jeweiligen Fahrzeugklassen dargestellt.
Die Annahmen wurden dem Szenario \textit{Verstärkte Elektrifizierung} für das Jahr \num{2049} aus der Studie \textit{Automobile Wertschöpfung 2030/2050} \cite{Kaul2019} entnommen.
Eine Limitierung der Ladeleistung durch die Unterscheidung in Normal- und Schnellladung erfolgt im Simulationsmodell von Seiten der Ladeinfrastruktur.

{
\renewcommand{\arraystretch}{1.2}% grßerer Zeilenabstand
\sisetup{range-phrase=~{--}~}% Gedankenstrich statt "bis" bei SIrange
\begin{table}[H]
	\begin{center}
		\caption{Maximale Ladeleistung und nutzbare Batteriekapazität je Fahrzeugklasse}
		\begin{tabu} to \textwidth {X[0.9] X[1.3, r] X[1.5, r]}
				\hline
				Fahrzeugklasse    & Maximale Ladeleistung in \si{\kw} & Nutzbare Batteriekapazität in \si{\kwh} \\ \hline
				BEV Kleinwagen    & \num{120}                         & \num{70}                                \\
				BEV Mittelklasse  & \num{350}                         & \num{100}                               \\
				BEV Oberklasse    & \num{350}                         & \num{120}                               \\
				PHEV Kleinwagen   & \num{120}                         & \num{25}                                \\
				PHEV Mittelklasse & \num{120}                         & \num{30}                                \\
				PHEV Oberklasse   & \num{120}                         & \num{40}                                \\ \hline
            \multicolumn{3}{l}{Quelle: Szenario \textit{Verstärkte Elektrifizierung} für das Jahr \num{2049} \cite{Kaul2019}}
		\end{tabu}
		\label{tab:TechPowerCap}
	\end{center}
	\vspace{-3mm}%Put here to reduce too much white space after your table
\end{table}
}

Der elektrische Energieverbrauch der \gls{EPKW} wird aus den Annahmen der Studie \textit{eMobil 2050} \cite{Hacker2014} abgeleitet.
Es wird angenommen, dass Kleinwagen gegenüber Mittelklasse-\gls{EPKW} einen um \SI{20}{\percent} reduzierten Energieverbrauch aufweisen.
Oberklasse-\gls{EPKW} weisen hingegen einen um \SI{20}{\percent} erhöhten Energieverbrauch auf.

Weiterhin bietet die Studie \textit{eMobil 2050} nur Verbrauchsangaben nach dem \glspl{NEFZ}, welche nicht realen Verbrauchsdaten entsprechen.
Nach \textit{Informationen zur Umweltpolitik - 189} \cite{Heinfellner2015} lag der Realverbrauch \num{2013} gegenüber einer Messung nach \gls{NEFZ} im Mittel um \SI{27}{\percent} höher.
Die Werte für das Jahr \num{2050} der Studie \textit{eMobil 2050} werden um diesen Faktor erhöht und die Ergebnisse in \autoref{tab:TechVerbrauch} zusammengefasst.

{
\renewcommand{\arraystretch}{1.2}% grßerer Zeilenabstand
\sisetup{range-phrase=~{--}~}% Gedankenstrich statt "bis" bei SIrange
\begin{table}[H]
	\begin{center}
		\caption{Durchschnittlicher elektrischer Energieverbrauch je Fahrzeugklasse}
		\begin{tabu} to 0.6\textwidth {X[1] X[1.2, r]}
			\toprule
			Fahrzeugklasse    & Verbrauch in \si{\kwhkm} \\ \midrule
			BEV Kleinwagen    & \num{11.9}              \\
			BEV Mittelklasse  & \num{14.8}              \\
			BEV Oberklasse    & \num{17.8}              \\
			PHEV Kleinwagen   & \num{12.1}              \\
			PHEV Mittelklasse & \num{15.2}              \\
			PHEV Oberklasse   & \num{18.2}              \\ \bottomrule
		\end{tabu}
		\label{tab:TechVerbrauch}
	\end{center}
	\vspace{-3mm}%Put here to reduce too much white space after your table
\end{table}
}


\paragraph{Ladeinfrastruktur:}

Die Ladeinfrastruktur wird im Gegensatz zu den Annahmen zum Fahrzeughochlauf nicht in absoluten Zahlen ausgedrückt.
Stattdessen werden den einzelnen Wegezwecken der \gls{EPKW} Wahrscheinlichkeiten zugeordnet, ob ein \gls{EPKW} am Zielort geladen werden kann.
Weiterhin wird diese Wahrscheinlichkeit auf verschiedene Ladeleistungen aufgeteilt.
Dabei wird grundsätzlich zwischen Normal- und Schnellladung unterschieden.


\subparagraph{Normalladung} beinhaltet in dieser Arbeit die Leistungsklassen \SI{3.7}{\kw}, \SI{11}{\kw}, \SI{22}{\kw} und \SI{50}{\kw}.
Um zu bestimmen, mit welcher Ladeleistung an einem Zielort geladen werden kann, werden den unterschiedlichen Wegezwecken nach \glsipl{MID} (s. \autoref{chap:MID}) dezidiert Wahrscheinlichkeiten zugeordnet, ob und mit welcher Ladeleistung geladen werden kann.
Eine Zuordnung der Leistungsklassen auf die einzelnen Wegezwecke erfordert vorerst eine Zuordnung der \UCs auf die Wegezwecke.
Unter den \UCs werden die fünf Standortarten \Eigenheimdot, \Wohnanlagedot, \Firmeparkplatzdot, \Gewerbeparkplatz und \Straszenrand verstanden.
In \autoref{tab:WegLadeUseCase} ist die entsprechende prozentuale Aufteilung der \UCs auf die Wegezwecke dargestellt.

{
\renewcommand{\arraystretch}{1.2}% grßerer Zeilenabstand
\sisetup{range-phrase=~{--}~}% Gedankenstrich statt "bis" bei SIrange
\begin{table}[H]
	\begin{center}
		\caption{Prozentuale Zuordnung der Lade Use cases auf die verschiedenen Wegezwecke}
		\begin{tabu} to \textwidth {X[1.2] X[1.1, r] X[1.3, r] X[1.7, r] X[1.8, r] X[1.2, r]}
			\toprule
			Wegezweck  & Eigenheim         & Wohnanlage        & Firmenparkplatz   & Gewerbeparkplatz  & Straßenrand       \\ \midrule
			Arbeit     & \SI{0}{\percent}  & \SI{0}{\percent}  & \SI{65}{\percent} & \SI{0}{\percent}  & \SI{35}{\percent} \\
			dienstlich & \SI{0}{\percent}  & \SI{0}{\percent}  & \SI{38}{\percent} & \SI{6}{\percent}  & \SI{56}{\percent} \\
			Ausbildung & \SI{0}{\percent}  & \SI{0}{\percent}  & \SI{65}{\percent} & \SI{0}{\percent}  & \SI{35}{\percent} \\
			Einkauf    & \SI{0}{\percent}  & \SI{0}{\percent}  & \SI{0}{\percent}  & \SI{77}{\percent} & \SI{24}{\percent} \\
			Erledigung & \SI{0}{\percent}  & \SI{0}{\percent}  & \SI{0}{\percent}  & \SI{38}{\percent} & \SI{62}{\percent} \\
			Freizeit   & \SI{0}{\percent}  & \SI{0}{\percent}  & \SI{0}{\percent}  & \SI{39}{\percent} & \SI{62}{\percent} \\
			nach Hause & \SI{43}{\percent} & \SI{25}{\percent} & \SI{0}{\percent}  & \SI{0}{\percent}  & \SI{31}{\percent} \\ \bottomrule
		\end{tabu}
		\label{tab:WegLadeUseCase}
	\end{center}
	\vspace{-3mm}%Put here to reduce too much white space after your table
\end{table}
}

Derzeit leben in Deutschland ungefähr \SI{44.2}{\MioMen} in Gebäuden mit maximal zwei Wohnungen und \num{37.2} Millionen Menschen in Mehrfamilienhäusern.
Etwa \SI{80}{\percent} der \linebreak Fahrzeugbesitzer$^*$innen in Gebäuden mit maximal zwei Wohnungen verfügen über einen Stellplatz in einer Garage oder unter einem Carport.
In Mehrfamilienhäusern verfügen hingegen nur \SI{55}{\percent} der Fahrzeugbesitzer$^*$innen über einen Stellplatz für ihr Fahrzeug \cite{dena2020}.
Unter der vereinfachenden Annahme einer gleichmäßigen Verteilung von Fahrzeugen zwischen Fahrzeugbesitzern$^*$innen in Ein- und Mehrfamilienhäusern ergibt sich hieraus die ermittelte Aufteilung der \UCs auf den Wegezweck \nHdot.\medskip

Die Aufteilung der \UCs auf die Wegezwecke \Einkaufdot, \Erledigung und \Freizeit ergeben sich aus den Wegeanteilen je Fahrtzweck im \gls{MIV}-Privatverkehr \cite{Rikus2015}.
Im Falle des Wegezwecks \Einkauf wird angenommen, dass Lebensmittelgeschäfte einen Gewerbeparkplatz für jeden Kunden bereitstellen.
Weiterhin wird angenommen, dass im Falle von sonstigen Waren und sonstigen Dienstleistungen in \SI{50}{\percent} der Fälle ein Gewerbeparkplatz zur Verfügung steht.
Bei dem Wegezweck \Erledigung wird davon ausgegangen, dass beim Besuch von Behörden, Banken, Post und Geldautomaten ein Gewerbeparkplatz vorhanden ist, während bei sonstigen Erledigungen dies nur in \SI{50}{\percent} der Fälle gegeben ist.
Für den Wegezweck \Freizeit wird angenommen, dass bei kulturellen Einrichtungen und Veranstaltungen ein Gewerbeparkplatz vorhanden ist.
Bei sonstigen Freizeitaktivitäten wird unterstellt, dass in \SI{50}{\percent} der Fälle ein Gewerbeparkplatz vorhanden ist.
In allen verbleibenden Fällen erfolgt ein Parken am Straßenrand.\medskip

Für die Abschätzung der Wahrscheinlichkeit, auf einem Firmenparkplatz für den Wegezweck \Arbeit parken zu können, wurde die Parkplatzsituation am Arbeitsplatz zugrunde gelegt.
Demnach werden \SI{67}{\percent} aller Arbeitswege mit dem \gls{PKW} zurückgelegt, wenn die Parkplatzsituation am Arbeitsplatz als nicht schwierig eingestuft wird und nur \SI{36}{\percent} bei einer schwierigen Parkplatzsituation.
Insgesamt werden mit dem \gls{PKW} \SI{56}{\percent} aller Arbeitswege zurückgelegt \cite{Ecke2020}.
Unter der Annahme, dass eine nicht schwierige Parkplatzsituation am Arbeitsplatz gleichbedeutend mit einem Firmenparkplatz und eine schwierige Parkplatzsituation mit dem Parken am Straßenrand ist, ergeben sich hieraus die ermittelten Anteile für den Wegezweck \Arbeitdot.
Weiterhin wird angenommen, dass dieses Verhältnis auf den Wegezweck \Ausbildung übertragen werden kann.\medskip

Für den Wegezweck \dienst wird die Aufteilung nach dem üblichen Stellplatz am Fahrtziel im Wirtschaftsverkehr verwendet \cite{Rikus2015}.
Demnach parken gewerbliche Halter im Wirtschaftsverkehr in \SI{30}{\percent} der Fälle am Straßenrand und in \SI{26}{\percent} der Fälle auf einem Privatgrundstück.
Im Wirtschaftsverkehr handelt es sich in der Regel nicht um das eigene Privatgrundstück, weshalb diese Art des Parkens ebenfalls als Parken am Straßenrand gewertet wird.
In \SI{6}{\percent} der Fälle erfolgt das Parken auf einem Gewerbeparkplatz und in \SI{38}{\percent} der Fälle auf einem Firmenparkplatz.\medskip

Im Folgenden erfolgt je \UC eine Abschätzung der Wahrscheinlichkeiten, inwieweit eine Ladung des \gls{EPKW} stattfinden und mit welcher Ladeleistung geladen werden kann.
In \autoref{tab:UCProbability2050} werden die entsprechenden Annahmen dargestellt.

{
\renewcommand{\arraystretch}{1.2}% grßerer Zeilenabstand
\sisetup{range-phrase=~{--}~}% Gedankenstrich statt "bis" bei SIrange
\begin{table}[H]
	\begin{center}
		\caption{Wahrscheinlichkeitverteilung der Ladeleistungen je Lade Use case}
		\begin{tabu} to \textwidth {X[1.7] X[1.3, r] X[1, r] X[1, r] X[1, r] X[1, r]}
			\hline
			Lade Use   Case  & keine Ladung      & \SI{3.7}{\kw}    & \SI{11}{\kw}      & \SI{22}{\kw}      & \SI{50}{\kw}      \\ \hline
			Eigenheim        & \SI{8}{\percent}  & \SI{0}{\percent} & \SI{69}{\percent} & \SI{23}{\percent} & \SI{0}{\percent}  \\
			Wohnanlage       & \SI{25}{\percent} & \SI{8}{\percent} & \SI{60}{\percent} & \SI{8}{\percent}  & \SI{0}{\percent}  \\
			Firmenparkplatz  & \SI{25}{\percent} & \SI{0}{\percent} & \SI{30}{\percent} & \SI{30}{\percent} & \SI{15}{\percent} \\
			Gewerbeparkplatz & \SI{25}{\percent} & \SI{0}{\percent} & \SI{8}{\percent}  & \SI{45}{\percent} & \SI{23}{\percent} \\
			Straßenrand      & \SI{75}{\percent} & \SI{0}{\percent} & \SI{10}{\percent} & \SI{10}{\percent} & \SI{5}{\percent}  \\ \hline
		\end{tabu}
		\label{tab:UCProbability2050}
	\end{center}
	\vspace{-3mm}%Put here to reduce too much white space after your table
\end{table}
}

In allen \UCs wird davon ausgegangen, dass die Ladevorgänge in Zukunft mit immer höheren Ladeleistungen stattfinden.
Für den \UC \Eigenheim wird angenommen, dass Besitzer\(^*\)innen eines \glspl{EPKW} eine Ladevorrichtung einrichtet, wenn die technischen Voraussetzungen gegeben sind.
Bei rund \SI{15}{\percent} der Stellplätze von Gebäuden mit einer oder zwei Wohnungen besteht kein Zugang zum Stromnetz \cite{dena2020}.
Es wird angenommen, dass von diesem Bestand die Hälfte einen Zugang zum Stromnetz erhält.
Die Aufteilung der Ladeleistungen erfolgt hierbei in Anlehnung an \textit{Bedarfsgerechte und wirtschaftliche öffentliche Ladeinfrastruktur} \cite{NPZMAVE2020}.
Es wird davon ausgegangen, dass der Anteil an Ladevorgängen mit einer Leistung von \SI{3.7}{\kw} nicht weiter relevant ist.
Der Anteil an Ladevorgängen mit \SI{11}{\kw} wird weiter wachsen, allerdings langsamer als der Anteil an Ladevorgängen mit \SI{22}{\kw}.
Eine Ladung mit \SI{50}{\kw} wird im privaten Bereich ebenfalls keine Rolle spielen.
Das prozentuale Verhältnis zwischen den Ladeleistungen beträgt \(0:75:25:0\)\medskip

Stellplätze von Mehrfamilienhäusern besitzen in ungefähr \SI{50}{\percent} der Fälle Zugang zum Stromnetz \cite{dena2020}.
Es wird angenommen, dass bei der Hälfte der Stellplätze ohne Zugang zum Stromnetz dieser nachgerüstet wird.
Abschließend wird angenommen, dass alle Stellplätze von Besitzer$^*$innen eines \glspl{EPKW} mit den entsprechenden technischen Voraussetzungen mit einer Ladevorrichtung ausgestattet werden.
In Wohnanlagen werden hohe Ladeleistungen eine Ausnahme bleiben.
Insgesamt wird von einem Verhältnis zwischen den Ladeleistungen von \(10:80:10:0\) ausgegangen.\medskip

Bei Firmen- bzw. Gewerbeparkplätzen wird unterstellt, dass zukünftig Mitarbeiter$^*$innen bzw. Kund$^*$innen in \SI{75}{\percent} der Fälle Zugriff auf einen Ladepunkt haben.
Die Verteilung der Ladeleistungen geschieht in Anlehnung an das Ladesäulenregister der Bundesnetzagentur \cite[][Stand: \DTMdate{2020-09-09}]{BundesnetzagenturElektrizitaet2020} und der Stromtankstellen Statistik des \textit{GoingElectric} Forums \cite[][Stand: \DTMdate{2020-10-21}]{Weemaes2020}.
Demnach besitzt ein Großteil (\SIrange[range-phrase=~bzw.~]{80}{51}{\percent}) der heute öffentlich zugänglichen Ladepunkte eine Ladeleistung von \SIrange{22}{42}{\kw}.
Insgesamt wird davon ausgegangen, dass sich der Trend zu hohen Ladeleistungen weiter fortsetzt.
Dies gilt jedoch vor allem für Gewerbeparkplätze, da hohe Ladeleistungen und die Verfügbarkeit von Ladepunkten zur Kundenakquise genutzt werden.
Bei dem \UC \Firmeparkplatz wird von einem Verhältnis von \(0:40:40:20\) und bei dem \UC \Gewerbeparkplatz von \(0:10:60:30\) ausgegangen.\medskip

Im Falle des \UC \Straszenrand wird angenommen, dass zukünftig noch in \SI{75}{\percent} der Fälle kein Ladepunkt zur Verfügung steht.
Weiterhin entspricht das Verhältnis der Ladeleistungen dem des \UC \Firmeparkplatzdot.\medskip

Aus den zuvor getroffenen Annahmen können nun die Wahrscheinlichkeiten für die Ladevorgänge je Wegezweck berechnet werden.
Die entsprechenden Ergebnisse sind in \autoref{tab:WegezweckProbability2050} dargestellt.

{
\renewcommand{\arraystretch}{1.2}% grßerer Zeilenabstand
\sisetup{range-phrase=~{--}~}% Gedankenstrich statt "bis" bei SIrange
\begin{table}[H]
	\begin{center}
		\caption{Wahrscheinlichkeitverteilung der Ladeleistungen je Wegezweck}
		\begin{tabu} to \textwidth {X[1.2] X[1.2, r] X[1, r] X[1, r] X[1, r] X[1, r]}
			\toprule
			Wegezweck  & keine Ladung      & \SI{3.7}{\kw}    & \SI{11}{\kw}      & \SI{22}{\kw}      & \SI{50}{\kw}      \\ \midrule
			Arbeit     & \SI{43}{\percent} & \SI{0}{\percent} & \SI{23}{\percent} & \SI{23}{\percent} & \SI{11}{\percent} \\
			dienstlich & \SI{53}{\percent} & \SI{0}{\percent} & \SI{17}{\percent} & \SI{20}{\percent} & \SI{10}{\percent} \\
			Ausbildung & \SI{43}{\percent} & \SI{0}{\percent} & \SI{23}{\percent} & \SI{23}{\percent} & \SI{11}{\percent} \\
			Einkauf    & \SI{37}{\percent} & \SI{0}{\percent} & \SI{8}{\percent}  & \SI{37}{\percent} & \SI{18}{\percent} \\
			Erledigung & \SI{56}{\percent} & \SI{0}{\percent} & \SI{9}{\percent}  & \SI{23}{\percent} & \SI{12}{\percent} \\
			Freizeit   & \SI{56}{\percent} & \SI{0}{\percent} & \SI{9}{\percent}  & \SI{23}{\percent} & \SI{12}{\percent} \\
			nach Hause & \SI{33}{\percent} & \SI{2}{\percent} & \SI{48}{\percent} & \SI{15}{\percent} & \SI{2}{\percent}  \\ \bottomrule
		\end{tabu}
		\label{tab:WegezweckProbability2050}
	\end{center}
	\vspace{-3mm}%Put here to reduce too much white space after your table
\end{table}
}


\subparagraph{Schnellladung} entspricht in dieser Arbeit einer Art Notfallladung.
Fällt der \gls{SOC} eines \gls{EPKW} unter \SI{20}{\percent}, wird eine Schnellladestation angefahren und der \gls{EPKW} für \SI{15}{\Minuten} geladen.
Bei Schnellladevorgängen wird in dieser Arbeit grundsätzlich zwischen einer Ladung mit \SI{150}{\kw} und mit \SI{350}{\kw} unterschieden.
Nach dem Ladesäulenregister der Bundesnetzagentur \cite[][Stand: \DTMdate{2020-09-09}]{BundesnetzagenturElektrizitaet2020} weisen bereits heute \SI{59}{\percent} der Ladeinfrastruktur mit einer Leistung von mehr als \SI{50}{\kw} eine Ladeleistung von über \SI{150}{\kw} auf.
Es ist davon auszugehen, dass sich auch bei der Schnellladeinfrastruktur der Trend zu hohen Ladeleistungen fortsetzt.
Weiterhin ist davon auszugehen, dass dies vor allem für Tankstellen außerhalb von Ortschaften gilt, da sich diese häufig an Autobahnen befinden.
Auf diesen werden vermehrt lange Strecken zurückgelegt, wodurch der Ladebedarf durch Schnellladeinfrastruktur besonders hoch ausfällt.
Innerhalb von Ortschaften ist eine kleinere Ladeleistung von \SI{150}{\kw} oftmals ausreichend.
In der Simulation erfolgt anhand der Distanz der vorangegangenen Fahrt eine Festlegung, inwieweit es sich um eine Tankstelle innerorts oder außerorts handelt.
Bei einer Strecke von \(>\!\SI{50}{\km}\) wird davon ausgegangen, dass sich die angefahrene Tankstelle außerorts befindet.
In \autoref{tab:SchnellProbability2050} sind die getroffenen Annahmen für die Schnellladeinfrastruktur dargestellt.

{
\renewcommand{\arraystretch}{1.2}% grßerer Zeilenabstand
\sisetup{range-phrase=~{--}~}% Gedankenstrich statt "bis" bei SIrange
\begin{table}[H]
	\begin{center}
		\caption{Wahrscheinlichkeitsverteilung der Ladeleistungen von Schnellladeinfrastruktur}
		\begin{tabu} to 0.5\textwidth {X[2] X[1, r] X[1, r]}
			\toprule
			Lade Use   Case      & \SI{150}{\kw}       & \SI{350}{\kw}       \\ \midrule
			Tankstelle innerorts & \SI{80}{\percent} & \SI{20}{\percent} \\
			Tankstelle außerorts & \SI{0}{\percent}  & \SI{100}{\percent} \\ \bottomrule
		\end{tabu}
		\label{tab:SchnellProbability2050}
	\end{center}
	\vspace{-3mm}%Put here to reduce too much white space after your table
\end{table}
}


\subsection{Sensitivität Firmenparkplatz}

Zusätzlich zu den zuvor beschriebenen Szenarien soll in dieser Arbeit eine Sensitivität in Form einer Szenarette untersucht werden.
Hierbei analysiert die \SzeFirmenparkplatz den Einfluss eines geringeren Anteils an Ladevorgängen des \UC \Firmeparkplatzdot.
Aufgrund der hohen Gleichzeitigkeit und der zeitlichen Überschneidungen der Standzeiten mit Zeiten einer hohen Erzeugung aus \glspl{PVA} kommt dem \UC \Firmeparkplatz eine besondere Bedeutung zu.
Die \SzeFirmenparkplatz soll bei ansonsten gleichbleibenden Eingangsparametern die Sensitivität des Antriebswende-Szenarios auf die Verfügbarkeit von Ladepunkten des \UC \Firmeparkplatz analysieren.
Hierfür wird angenommen, dass nur noch in \SI{50}{\percent} der Fälle eine Ladung beim \UC \Firmeparkplatz stattfinden kann.

{
\renewcommand{\arraystretch}{1.2}% grßerer Zeilenabstand
\sisetup{range-phrase=~{--}~}% Gedankenstrich statt "bis" bei SIrange
\begin{table}[H]
	\begin{center}
		\caption{Anpassung der Wahrscheinlichkeitsverteilung der Ladeleistungen für die \SzeFirmenparkplatzdot}
		\begin{tabu} to \textwidth {X[1.7] X[1.3, r] X[1, r] X[1, r] X[1, r] X[1, r]}
			\toprule
			Lade Use Case   & keine Ladung        & \SI{3.7}{\kw}      & \SI{11}{\kw}        & \SI{22}{\kw}        & \SI{50}{\kw}        \\ \midrule
			Firmenparkplatz & \SI{50}{\percent} & \SI{0}{\percent} & \SI{20}{\percent} & \SI{20}{\percent} & \SI{10}{\percent} \\ \bottomrule
		\end{tabu}
		\label{tab:UCProbabilitySzenarette}
	\end{center}
	\vspace{-3mm}%Put here to reduce too much white space after your table
\end{table}
}

In \autoref{tab:UCProbabilitySzenarette} ist die hieraus entstehende Wahrscheinlichkeitsverteilung für den \UC \Firmeparkplatz dargestellt.
Hierdurch kommt es für die Wegezwecke \Arbeitdot, \dienst und \Ausbildung zu einer veränderten Wahrscheinlichkeitsverteilung, welche in \autoref{tab:WegezweckProbabilitySzenarette} dargestellt ist.

{
\renewcommand{\arraystretch}{1.2}% grßerer Zeilenabstand
\sisetup{range-phrase=~{--}~}% Gedankenstrich statt "bis" bei SIrange
\begin{table}[H]
	\begin{center}
		\caption{Anpassung der Wahrscheinlichkeitverteilung der Ladeleistungen je Wegezweck für die \SzeFirmenparkplatzdot}
		\begin{tabu} to \textwidth {X[1.2] X[1.2, r] X[1, r] X[1, r] X[1, r] X[1, r]}
			\hline
			Wegezweck  & keine Ladung      & \SI{3.7}{\kw}    & \SI{11}{\kw}      & \SI{22}{\kw}      & \SI{50}{\kw}     \\ \hline
			Arbeit     & \SI{59}{\percent} & \SI{0}{\percent} & \SI{16}{\percent} & \SI{16}{\percent} & \SI{8}{\percent} \\
			dienstlich & \SI{63}{\percent} & \SI{0}{\percent} & \SI{14}{\percent} & \SI{16}{\percent} & \SI{8}{\percent} \\
			Ausbildung & \SI{59}{\percent} & \SI{0}{\percent} & \SI{16}{\percent} & \SI{16}{\percent} & \SI{8}{\percent} \\ \hline
		\end{tabu}
		\label{tab:WegezweckProbabilitySzenarette}
	\end{center}
	\vspace{-3mm}%Put here to reduce too much white space after your table
\end{table}
}


\clearpage
