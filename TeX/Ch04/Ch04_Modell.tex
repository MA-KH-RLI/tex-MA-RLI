% Eingangsdaten der Simulation

\section{Modellbildung}

Grundlage für die Lastflussberechnung durch \edisgo bilden verschiedene Modelleingangsdaten.
Hierzu gehören in erster Linie die mit \simbev erzeugten Lastprofile der Ladevorgänge der \gls{EPKW}, die Topologie der \gls{MS}- und \gls{NS}-Netze, sowie die Erzeugungsprofile erneuerbarer Energieanlagen in einem Netzgebiet.
Weiterhin gehen auch die Lastprofile von Wärmepumpen in die Lastflussberechnung mit ein.
Dieses Kapitel soll einen Überblick über das verwendete Modell und die Eingangsdaten geben.\medskip


\subsection{Erzeugung der Lastzeitreihen der Ladevorgänge von E-Pkw}

% TODO: ding0

In diesem Kapitel wird auf die theoretischen Grundlagen und die simulative Erzeugung der Lastzeitreihen von \glspl{EPKW} eingegangen.
Die Lastzeitreihen der \glspl{EPKW} und die Optimierung dieser hin zu einem möglichst netzfreundlichen Verhalten, bilden den flexiblen Anteil der Eingangsdaten für die Lastflussberechnung mit \edisgo.\medskip

Mit Hilfe des im Rahmen dieser Masterarbeit mitentwickelten Software Tools \simbev können die Fahrtprofile für eine beliebige Anzahl an Fahrzeugen der verschiedenen Klassen für verschiedene Raumtypen erstellt werden (s. \autoref{chap:simbev_theo}.
Die Fahrtprofile enthalten die gefahrenen Strecken und die Standzeiten am Zielort.
Aus diesen können anschließend anhand des Verbrauchs und der gegebenen Ladeinfrastruktur die Ladebedarfe am Zielort abgeleitet werden.
Abhängig von der Ladestrategie wird innerhalb der Standzeit die Last auf die Standzeit verteilt.
Die Zeitreihen der Ladelast der einzelnen Fahrzeuge werden innerhalb eines Landkreises berechnet und schlussendlich zu einer Gesamtlast je \UC zusammengeführt.
Mit Hilfe des proprietär \localiserToolsKomma , werden die Zeitreihen der Gesamtlasten auf konkrete georeferenzierte Ladepunkte innerhalb des Landkreises verteilt.
Um abschließend eine Lasflussrechnung eines Netzgebietes mit \edisgo durchführen zu können, muss die Schnittmenge eines Netzgebietes mit den Landkreisen bestimmt werden.
Ladepunkte die innerhalb dieser Schnittmenge liegen, werden dem jeweiligen Netzgebiet zugeordnet und somit die entsprechende Last der Ladevorgänge.
Da die Simulation aller Netzgebiete zu inakzeptabel hohen Rechenzeiten führt, werden die \dingo Netzgebiete vorerst geclustert und Referenznetzgebiete identifiziert.


\subsubsection{Clustering der ding0 Netzgebiete}

% TODO: Auswertungen von Birgit übernehmen
% TODO: Karte mit den Netzgebieten
% TODO: welche charakteristischen Attribute wurden verwendet? @Birgit
% TODO: Schnittmenge Gemeinden <-> Netzgebiet erklären

Die Untersuchung der mit Hilfe des Software Tools \dingo synthetisierten \gls{MS}-Netzgebiete, führt aufgrund ihrer großen Anzahl zu inakzeptabel hohen Rechenzeiten während der Optimierung und Simulation.
Im Rahmen dieser Masterarbeit wurden sechs Referenznetzgebiete ausgewählt, welche einen Großteil der \num{3591} \gls{MS}-Netze repräsentieren.\medskip

Das Clustering erfolgt durch den während des \openego Projekt entwickelten k-means-Clusteralgorithmus (s. \autoref{chap:dingo_theo}).
Hierfür wurden die kumulierte Wind- und kumulierte Photovoltaikkapazität, sowie die kumulierte Last im Netzgebiet als charakteristische Attribute der \gls{MS}-Netze festgelegt


\subsubsection{Erstellung und Bewertung der Fahrtprofile mit simBEV}

Mit Hilfe des Software Tools \simbev (s. \autoref{chap:simbev_theo}) können für die zuvor geclusterten Referenznetzgebiet die entsprechenden Fahrtprofile der im Netzgebiet befindlichen Fahrzeuge erzeugt werden.
Um die Anzahl an Fahrzeugen je Netzgebiet zu bestimmen, muss der Gesamtbestand an Fahrzeugen je Szenario (s. \autoref{tab:SzenarienRampUp} und \autoref{tab:CarSplit}) regionalisiert werden.
Die Regionalisierung der Fahrzeuge findet vorerst auf Ebene der Landkreise statt.
Als Grundlage hierfür dient der aktuelle Fahrzeugbestand nach Zulassungsbezirken \cite[][Stand: \DTMdate{2020-01-01}]{KBAPLZ2020}.
Es wird davon ausgegangen, dass es zu keiner Verschiebung des Anteils am Bestand zwischen den Zulassungsbezirken kommt.
Dies bedeutet, dass die Gesamtanzahl der Fahrzeuge je Fahrzeugklasse je Szenario entsprechend des heutigen Bestandes anteilig verteilt wird.
Die Aufteilung der Fahrzeuge in Klassen erfolgt anhand der Einteilung des Fahrzeugsbestands in Hubraum-Klassen, welche ebenfalls dem Fahrzeugbestand nach Zulassungsbezirken entnommen werden können.\medskip

Die Einteilung der \Regiostar Raumtypen erfolgt auf Gemeindeebene, weshalb eine weitere Regionalisierung der Fahrzeuge innerhalb eines Landkreises auf die jeweiligen Gemeinden nötig ist.
Da auf Gemeindeebene keine Daten zum Fahrzeugbestand vorliegen, erfolgt die Regionalisierung anhand der Einwohnerzahl der Gemeinden.
Die Grundlage hierfür bildet der Datensatz \glqq Gemeindegrenzen 2017 mit Einwohnerzahl\grqq{} \cite[][Stand: \DTMdate{2017-12-31}]{EDG2020}.
Die Verteilung der Anzahl der Fahrzeuge erfolgt streng proportional zur Einwohnerzahl in der jeweiligen Gemeinde.\medskip

Um abschließend die Fahrtprofile erzeugen zu können, muss jeder Gemeinde ein \Regiostar Raumtyp zugeordnet werden.
Die entsprechende Zuordnung für das Jahr \num{2018} kann dem Datensatz \glqq Referenzdateien zur regionalstatistischen Raumtypologie\grqq{} \cite[][Stand: \DTMdate{2018-01-01}]{BMVIa2020} entnommen werden.\medskip

% TODO: Lastprofil und Dauerlastkurve erstellen für ein Beispiel
% Kritik: zu wenig HPC und nur für eine Woche möglich


\subsubsection{Implementierung der Ladestrategien in simbev}

Bereits ohne 


\subsubsection{Localiser Tool}


\subsection{eDisGo Lastflussberechnung}