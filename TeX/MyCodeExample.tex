\begin{code}
    \captionof{listing}{Gaußsches Eliminationsverfahren}\label{code:GaussElem}
    \begin{minted}[
		linenos=true,
		numbersep=-10pt,
		obeytabs=true,
		tabsize=4,
		]{matlab}
		%% Gaussian Elimination
		% bringing the Matrix into upper triangular form
		% initializing a new matrix to keep the previous results
		GaussMat = OrdMat;
		% and a new vector
		GaussVec = OrdVec;
		% for loop over m-1 elements
		for i = 1:m-1
		   % Calculating the Row Multiplier by dividing the
		   % i Element of the row i+1:m through the
		   % i,i Element of the matrix
		   Mult = GaussMat(i+1:m,i) / GaussMat(i,i);
		   % Calculating the new rows of the matrix
		   GaussMat(i+1:m,:) = GaussMat(i+1:m,:) - Mult*GaussMat(i,:);
		   % and the vector
		   GaussVec(i+1:m,:) = GaussVec(i+1:m,:) - Mult*GaussVec(i,:);
		end

		% removing rounding errors near zero
		threshold=10^(-12);
		% set near zeros to zero in the Matrix
		GaussMat(abs(GaussMat)<threshold) = 0;
		% set near zeros to zero in the Vector
		GaussVec(abs(GaussVec)<threshold) = 0;
		% initializing the solution vector x
		X = zeros(m,1);
		% determining the first element of the solution
		X(m,:) = GaussVec(m,:)/GaussMat(m,m);
		% for loop over the missing solutions
		for i = m-1:-1:1
		   % Calculating the i element of x by substracting product
		   % of the known elements of x and the corresponding row of
		   % the Gauss Matrix from the corresponding element of the
		   % Gauss Vector and dividing the outcome by
		   % the matrix element i,i
		   X(i,:) = ((GaussVec(i,:) -...
					GaussMat(i,i+1:m)*X(i+1:m,:))/GaussMat(i,i));
		end
	\end{minted}
\end{code}