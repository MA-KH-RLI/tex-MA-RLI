\selectlanguage{english} 

\begin{abstract}

\end{abstract}

\selectlanguage{ngerman}

\begin{abstract}
	
	Das Ziel der Dekarbonisierung des Verkehrssektors macht eine rapide Steigerung des Hochlaufs an direktelektrifizierten Elektrofahrzeugen aus heutiger Sicht unumgänglich.
	Als Konsequenz ist mit einer Zunahme der negativen Auswirkungen auf die Verteilnetze zu rechnen.
	Aus diesem Grund wird in dieser Arbeit untersucht, ob mit Hilfe von netzdienlichen Ladestrategien die Netzintegration von E-Pkw und erweiternd von fluktuierenden Erneuerbaren Energien unterstützt werden kann.
	Hierbei werden zwei präventive und eine aktive Ladestrategie auf ihre Wirksamkeit geprüft, um einen Vergleich zwischen den Ansätzen zu ermöglichen.
	Mit \mbox{\textit{simBEV}} wird ein Tool für die Erstellung von Fahrtprofilen und der Ermittlung des Ladebedarfs von E-Pkw mitentwickelt.
	Der Bedarf wird örtlich allokiert und in Form von Lastzeitreihen in die räumlich und zeitlich hochaufgelösten Netzmodelle von fünf typischen Mittelspannungsnetzen inklusive darunterliegenden Niederspannungsnetzen überführt.
	Nachfolgend wird für die Ermittlung von etwaigen Netzproblemen eine Lastflussanalyse durchgeführt.
	Abschließend erfolgt eine Ermittlung des last- und erzeugerseitigen Abregelungsbedarfs, welcher notwendig ist, um die Netzprobleme aufzulösen.
	Anhand dieses Wertes lassen sich Aussagen darüber treffen, inwieweit die Ladestrategien dazu in der Lage sind, kritische Netzbelastungen zu vermeiden beziehungsweise zu reduzieren.
	Es zeigt sich, dass eine präventive Ladestrategie mit reduzierten Ladeleistungen erfolgreich die lastseitigen Belastungen der Netze senken kann.
	Allerdings führen die präventiven Ladestrategien auch zu einer Erhöhung des Abregelungsbedarfs von fluktuierenden Erneuerbaren Energien, da sich der Ladebedarf in Zeiten hoher Einspeisung reduziert.
	Bei der aktiven Ladestrategie, welche sich an der Residuallast im Netzgebiet orientiert, kann der lastseitige Abregelungsbedarf in einzelnen Fällen stärker, aber in den meisten Fällen weniger stark als bei der präventiven Ladestrategie mit reduzierten Ladeleistungen gesenkt werden.
	Auf der anderen Seite bietet vor allem in Photovoltaik-dominierten Netzen die aktive Ladestrategie als einzige Ladestrategie das Potential, den erzeugerseitigen Abregelungsbedarf zu senken und somit die Netzintegration von fluktuierenden Erneuerbaren Energien zu unterstützen.
	Der Erfolg der aktiven Ladestrategie hängt last- und erzeugerseitig davon ab, in welchem Maße die globale Residuallast im Mittelspannungsnetzgebiet die Situationen in den einzelnen Netzabschnitten widerspiegelt.
	So erweist sich primär in Wind-dominierten Netzen die globale Residuallast als unzureichende Optimierungsgröße, welches zu negativen Effekten sowohl auf den last- als auch erzeugerseitigen Abregelungsbedarf führen kann.
	
\end{abstract}

\clearpage