\selectlanguage{english} 

\begin{abstract}

\lipsum[12]

\end{abstract}

\selectlanguage{ngerman}

\begin{abstract}
	
	Das Ziel der Dekarbonisierung des Verkehrssektors macht eine rapide Steigerung des Hochlaufs an direktelektrifizierten Elektrofahrzeugen aus heutiger Sicht unumgänglich.
	Angesicht der hohen Ladeleistungen und der zu erwartenden Gleichzeitigkeiten von Ladevorgängen ist mit dem Hochlauf an Elektrofahrzeugen mit einer Zunahme von Rückwirkungen auf die Verteilnetze zu rechenen.
	Aus diesem Grund wird in dieser Arbeit untersucht, ob mit Hilfe von zwei präventiven und einer aktiven netzdienlichen Ladestrategie die Netzintegration von E-Pkw und erweiternd fluktuierenden erneuerbaren Energien unterstützt werden kann.
	Mit \mbox{\textit{simBEV}} wird ein Tool entwickelt zur Ermittlung des Ladebedarfs von E-Pkw und dessen zeitlicher Allokation.	
	Der Bedarf wird einer konkreten Ladeinfrastruktur zugewiesen und in Form von Lastzeitreihen in die räumlich und zeitlich hochaufgelösten Netzmodelle von fünf Referenznetzgebieten überführt.
	Nachfolgend wird für die Ermittlung von etwaigen Netzproblemen eine Lastflussanalyse durchgeführt.
	Um Abschließend einen Vergleich der Ladestrategien zu ermöglichen, wird der Abregelungsbedarf ermittelt, welcher nötig ist, um die Netzprobleme aufzulösen.
	Es zeigt sich, dass ausschließlich die aktive Ladestrategie, welche sich an der Residuallast im Netzgebiet orientiert, die Integration von fluktuierenden erneuerbaren Energien unterstützen kann.
	Lastseitig kann neben der aktiven Ladestrategie insbesondere durch eine präventive Ladestrategie mit reduzierten Ladeleistungen erfolgreich Abregelungsbedarf vermieden werden.
	Der Erfolg der aktiven Ladestrategie ist sowohl last- als auch erzeugerseitig Abhängig von einigen Randbedingungen und nicht garantiert.
	Vor allem in Wind-dominierten Netzen müssen zusätzliche Randbedingungen beachtet werden, um eine erfolgreiche aktive Ladestrategie zu gestalten.
	
\end{abstract}

\clearpage