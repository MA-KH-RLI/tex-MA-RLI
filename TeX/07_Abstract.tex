\selectlanguage{english} 

\begin{abstract}

\lipsum[12]

\end{abstract}

\selectlanguage{ngerman}

\begin{abstract}
	
	Das Ziel der Dekarbonisierung des Verkehrssektors macht eine rapide Steigerung des Hochlaufs an direktelektrifizierten Elektrofahrzeugen aus heutiger Sicht unumgänglich.
	Mit dem Hochlauf an Elektrofahrzeugen ist mit einer Zunahme der negativen Auswirkungen auf die Verteilnetze zu rechnen.
	Aus diesem Grund wird in dieser Arbeit untersucht, ob mit Hilfe von netzdienlichen Ladestrategie die Netzintegration von E-Pkw und erweiternd fluktuierenden Erneuerbaren Energien unterstützt werden kann.
	Hierbei werden zwei präventive und eine aktive Ladestrategie auf ihre Wirksamkeit geprüft, um eine Vergleich zwischen den Ansätzen zu ermöglichen.
	Mit \mbox{\textit{simBEV}} wird ein Tool entwickelt zur Ermittlung von Fahrtprofilen und des Ladebedarfs von E-Pkw.	
	Der Bedarf wird örtlich allokiert und in Form von Lastzeitreihen in die räumlich und zeitlich hochaufgelösten Netzmodelle von fünf typischen Mittelspannungsnetzen, inklusive darunterliegender Niederspannungsnetze, überführt.
	Nachfolgend wird für die Ermittlung von etwaigen Netzproblemen eine Lastflussanalyse durchgeführt.
	Um Abschließend einen Vergleich der Ladestrategien zu ermöglichen, wird der last- und erzeugerseitige Abregelungsbedarf ermittelt, welcher nötig ist, um die Netzprobleme aufzulösen.
	Es zeigt sich, dass ausschließlich die aktive Ladestrategie, welche sich an der Residuallast im Netzgebiet orientiert, die Integration von fluktuierenden Erneuerbaren Energien unterstützen kann.
	Lastseitig kann neben der aktiven Ladestrategie insbesondere durch eine präventive Ladestrategie mit reduzierten Ladeleistungen erfolgreich Abregelungsbedarf vermieden werden.
	Der Erfolg der aktiven Ladestrategie ist sowohl last- als auch erzeugerseitig abhängig von einigen Randbedingungen und nicht garantiert.
	Vor allem in Wind-dominierten Netzen müssen zusätzliche Randbedingungen beachtet werden, um eine erfolgreiche aktive Ladestrategie zu gestalten.
	
\end{abstract}

\clearpage