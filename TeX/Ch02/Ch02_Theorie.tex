\section{Theoretischer Hintergrund}

\subsection{Allgemeine Definitionen}

\paragraph{Elektrische Flexibilität:}

% s. Anyas Folien vom Workshop - räumlich und zeitlich

\paragraph{Spannungsebenen:}

Innerhalb der Verteilnetze wird grundlegend zwischen drei Spannungsebenen unterschieden. Hierzu zählen die \glspl{HS}-, \glspl{MS}- und \glspl{NS}-Ebene.

{
\renewcommand{\arraystretch}{1.2}% grßerer Zeilenabstand
\sisetup{range-phrase=~oder~}
\begin{table}[H]
	\begin{center}
		\caption{Übliche Spannung und Stromkreislänge der Spannungsebenen im deutschen Verteilnetz}
		\begin{tabu} to \textwidth {X[1] X[1, r] X[1, r]}
			\hline
            Spannungsebene & Spannung               & Stromkreislänge   \\\hline
            Hochspannung   & \SI{110}{\kv}          & \SI{95000}{\km}   \\
            Mittelspannung & \SIrange{20}{10}{\kv}  & \SI{510000}{\km}  \\
            Niederspannung & \SIrange{400}{230}{\V} & \SI{1100000}{\km} \\\hline
            \multicolumn{3}{l}{Quelle: \cite{BDEW2016}}
		\end{tabu}
		\label{tab:Spannungsebenen}
	\end{center}
	\vspace{-3mm}%Put here to reduce too much white space after your table
\end{table}
}

\paragraph{Netztopologie:}

% Strahlen- und Ringnetze

\paragraph{Gleichzeitigkeit:}

\subsection{Elektromobilität}

\paragraph{Ladestrategien:}

Der Ladevorgang von \glspl{EV} kann durch unterschiedliche äußere Anreize gesteuert werden. Grundsätzlich lassen sich hierbei marktorientierte und netzdienliche Ladestrategien unterscheiden.

%ToDo! systemorientiert Ladestrategien ?!?

\subparagraph{Marktorientierte Ladestrategien} haben als Fokus die Minimierung der Kosten für den Strombezug. Dies bedeutet konkret, dass die Ladevorgänge durch ein Preissignal am Großhandelsmarkt ausgelöst beziehungsweise unterbrochen werden. Eine solche Ladestrategie kann sowohl positive als auch negative Effekte aufweisen und erfordern einen geeigneten rechtlichen Rahmen. So führt beispielsweise ein hohes Stromangebot zu niedrigen Großhandelsmarktpreisen, wodurch das beladen der \glspl{EV} ausgelöst wird und ein Ausgleich zwischen Angebot und Nachfrage angestrebt wird. Auf der anderen Seite werden lokale Netzengpässe nicht berücksichtigt und die Gleichzeitigkeit der Ladevorgänge erhöht sich, wodurch sich der Netzausbaubedarf erhöhen kann. \cite{Agora2019} \cite{Dorendorf2019}

% BW Verteilnetzstudie: Bei einem marktorientierten Betrieb von Flexibilitätsoptionen richten sich diese in ihrem Betriebsverhalten nach den Anforderungen eines überregionalen Marktes.
% Als Ergebnis des Analyseschrittes ist bekannt, welches Betriebsverhalten die betrachteten Flexibilitätsoptionen bei einer marktorientierten Nutzung aufweisen können. Hierbei zeigt sich, dass die überregionalen Flexibilitätssignale eine hohe Gleichzeitigkeit im Betrieb insbesondere der lastseitigen Flexibilitätsoptionen auslösen. Besonders bei EV findet eine starke Konzentration der Ladevorgänge auf die frühen Morgenstunden statt (siehe Abbildung 6.6). 

\subparagraph{Netzdienliche Ladestrategien} setzen hingegen auf die Vermeidung von lokalen Engpässen, welche durch eine hohe Nachfrage entstehen können. Hierbei kann zwischen präventiven und kurative Maßnahmen unterschieden werden. Präventive Maßnahmen sollen Kunden dazu bewegen, ihre Ladevorgänge in Zeiten geringer Netzauslastung zu verlegen. Dies kann zum Beispiel über monetäre Anreize aber auch über Quoten erfolgen. Bei kurativen Maßnahmen handelt es sich hingegen um ein aktives Eingreifen durch den Netzbetreiber, welcher bei drohenden Netzengpässen in den Ladevorgang eingreift. \cite{Agora2019}

% präventiv, aktiv, kurativ

\subsection{eDisGo}

\paragraph{Netzausbaubedarf:}

% s. Verteilnetzstudie BW
% Die festgestellten Grenzwertverletzungen werden wie folgt priorisiert:
% 1. Verletzung der thermischen Transformatorbetriebsgrenzen
% 2. Verletzungen des Spannungsbandes
% 3. Verletzung der thermischen Leitungsbetriebsgrenzen

\subparagraph{Minimaler Residuallastfall und Maximaler Residuallastfall}

\subparagraph{Thermischen Betriebsmittelbelastung}

\subparagraph{Maximale Knotenspannungen}

% Spannungsband

\subparagraph{n-1}

\paragraph{Netzausbaumaßnahmen:}

% BW: In der klassischen Netzplanung erfolgt die Bestimmung der auslegungsrelevanten Betriebsfälle anhand der installierten Leistung einzelner Verbraucher, gemessenen Spitzenlasten oder auf Basis von Erfahrungswerten bezüglich der Gleichzeitigkeit bestimmter Verbraucher. Eine Netzplanung auf Basis von Zeitreihen findet in Verteilnetzen in der Regel nicht statt. Da die Analyse des Betriebsverhaltens von Flexibilitätsoptionen jedoch auf der Betrachtung von Zeitreihen aufbaut, ergibt sich das Problem der Übertragbarkeit von Erkenntnissen aus der Zeitreihenanalyse auf die Netzplanung.

\subparagraph{rONT}

% Ergebnis Workshop: i.d.R. zu teuer und unnötig

% BW: Eine häufige Ursache für Netzausbaubedarf bzw. auftretende restriktive Randbedingung für die Integration von DEA in NS-Netze ist die Einhaltung der Spannungsqualität beim Verbraucher gemäß DIN EN 50160 [16]. 

\subparagraph{Spannungsbandaufteilung}

\subparagraph{Optimierte Einstellung eines lastfrei stufbaren Transformators}