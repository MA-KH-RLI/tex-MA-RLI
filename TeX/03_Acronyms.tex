% Abkürzungen
% s. documentation https://ctan.org/pkg/glossaries?lang=de
% \newacronym[
% shortplural = {}, -> Hier kann der Plural der Kurzform eigens eingestellt werden
% longplural = {}, -> Hier kann der Plural der ausgeschriebenen Form eigens eingestellt werden
% {html} -> label
% {HTML} -> short form
% {hypertext markup language} -> long form

\makeglossaries

\newcommand{\glsi}[1]{\textit{\gls{#1}}} % italic gls
\newcommand{\glsipl}[1]{\textit{\glspl{#1}}} % italic glspl

%\setacronymstyle{long-short}
\newacronym[
	]{BEV}{BEV}{Battery electric vehicle}
\newacronym[
	shortplural={\textit{ding0s}},
	]{DINGO}{ding0}{DIstribution Network Generat0r}
\newacronym[
	longplural={electricity Distribution Grid optimization},
	shortplural={eDisGos},
	]{EDISGO}{eDisGo}{electricity Distribution Grid optimization}
\newacronym[
    longplural={elektrische Personenkraftwagen},
	shortplural={E-Pkw}
	]{EPKW}{E-Pkw}{elektrischer Personenkraftwagen}
\newacronym[
	longplural={fluktuierender Erneuerbarer Energien},
	shortplural={fEE}
	]{FEE}{fEE}{fluktuierende Erneuerbare Energien}
\newacronym[
	]{G2V}{G2V}{Grid-to-vehicle}
\newacronym[
	]{GHD}{GHD}{Gewerbe, Handel, Dienstleistungen}
\newacronym[
    shortplural={HS},
	]{HPC}{HPC}{High Power Charging}
\newacronym[
    shortplural={HS},
	]{HS}{HS}{Hochspannung}
\newacronym[
	longplural={Identifikatoren}
	]{ID}{ID}{Identifikator}
\newacronym[
	longplural={Mobilität in Deutschland},
    shortplural={MiD 2017},
	]{MID}{MiD 2017}{Mobilität in Deutschland}
\newacronym[
	longplural={motorisierten Individualverkehrs},
    shortplural={MIV},
	]{MIV}{MIV}{motorisierter Individualverkehr}
\newacronym[
	longplural={mittleren quadratischen Abweichung},
    shortplural={MQA},
	]{MQA}{MQA}{mittlere quadratische Abweichung}
\newacronym[
    shortplural={MS},
	]{MS}{MS}{Mittelspannung}
\newacronym[
	longplural={neuen europäischen Fahrzyklus},
    shortplural={NEFZ},
	]{NEFZ}{NEFZ}{Neuer Europäischer Fahrzyklus}
\newacronym[
	longplural={Netzentwicklungsplans},
    shortplural={NEP},
	]{NEP}{NEP}{Netzentwicklungsplan}
\newacronym[
    shortplural={NS},
	]{NS}{NS}{Niederspannung}
\newacronym[
    shortplural={OEP},
	]{OEP}{OEP}{Open Energy Platform}
\newacronym[
	longplural={Ortsnetzstationen},
    shortplural={ONS},
	]{ONS}{ONS}{Ortsnetzstation}
\newacronym[
	]{OPENEGO}{open\_eGo}{open electricity Grid optimization}
\newacronym[
    shortplural={OSM},
	]{OSM}{OSM}{OpenStreetMap}
\newacronym[
	sort={P},
	]{P}{$P$}{Wirkleistung}
\newacronym[
	]{PHEV}{PHEV}{Plug-in hybrid electric vehicle}
\newacronym[
    longplural={Personenkraftwagen},
	shortplural={Pkw},
	]{PKW}{Pkw}{Personenkraftwagen}
\newacronym[
    longplural={Points of interest},
	]{POI}{POI}{Point of interest}
\newacronym[
    longplural={Photovoltaik},
    shortplural={PV}
	]{PV}{PV}{Photovoltaik}
\newacronym[
    longplural={Photovoltaikanlagen},
    shortplural={PVA}
	]{PVA}{PVA}{Photovoltaikanlage}
\newacronym[
	sort=Q
	]{Q}{$Q$}{Blindleistung}
\newacronym[
    longplural={regionalstatistischen Raumtypologien 7},
    shortplural={RegioStaR 7}
	]{REGIOSTAR}{RegioStaR 7}{regionalstatistische Raumtypologie 7}
\newacronym[
	]{SIMBEV}{SimBEV}{Simulative Battery Electric Vehicle}
\newacronym[
    longplural={State of charge},
    shortplural={SOC}
	]{SOC}{SoC}{State of charge}
\newacronym[
    longplural={Umspannwerke},
    shortplural={USW},
	]{USW}{USW}{Umspannwerk}
\newacronym[
    shortplural={V2G},
	]{V2G}{V2G}{Vehicle-to-grid}
\newacronym[
    longplural={Wärmepumpen},
    shortplural={WP},
	]{WP}{WP}{Wärmepumpe}